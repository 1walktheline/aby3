\newif\iffullversion
\fullversionfalse
%\fullversiontrue

\documentclass[11pt,letterpaper]{article}
\usepackage[paper=letterpaper,margin=1in]{geometry}

\usepackage{amsmath,amsfonts,amssymb,tikz,multirow,stmaryrd,xspace}
\usepackage[hyphens]{url}
\usepackage{varwidth}
\usepackage{amsthm}
\usepackage{caption}

\usepackage{amssymb}
\usepackage{pgfplots}
\pgfplotsset{compat=newest}

%this should save a minimum amount of space and in theory look nicer
\usepackage{ellipsis, ragged2e, marginnote}
\usepackage[tracking=true]{microtype}
\DeclareMicrotypeSet*[tracking]{my}{ font = */*/*/sc/* }
\SetTracking{encoding = *, shape = sc}{45}

\usepackage[]{hyperref}

\hypersetup{
    colorlinks=true,
    linkcolor=darkred,
    citecolor=darkgreen,
    urlcolor=darkblue
}

\newcommand{\false}{\ensuremath{\textsc{false}}\xspace}
\newcommand{\true}{\ensuremath{\textsc{true}}\xspace}

\newcommand{\naive}{na\"{\i}ve\xspace}
\newcommand{\Naive}{Na\"{\i}ve\xspace}

\newcommand{\share}[1]{\ensuremath{\llbracket #1\rrbracket}\xspace}
\newcommand{\F}{\ensuremath{\mathbb{F}}\xspace}
\newcommand{\C}{\ensuremath{\mathcal{C}}\xspace}
\newcommand{\M}{\ensuremath{\mathcal{M}}\xspace}
\renewcommand{\O}{\ensuremath{\mathcal{O}}\xspace}
\renewcommand{\P}{\ensuremath{\mathcal{P}}\xspace}
\newcommand{\A}{\ensuremath{\mathcal{A}}\xspace}
\newcommand{\R}{\ensuremath{\mathcal{R}}\xspace}

\newcommand{\myvec}[1]{\boldsymbol{#1}}

\newcommand{\algorithm}[1]{\ensuremath{\text{\sf #1}}\xspace}

\newcommand{\DPFgen}{\algorithm{DPF.Gen}}
\newcommand{\DPFexpand}{\algorithm{DPF.Expand}}

\newcommand{\serv}[1]{server~\##1}
\newcommand{\Serv}[1]{Server~\##1}
\def\SSS{\ensuremath{\mathcal{S}}\xspace}
\def\RR{\ensuremath{\mathcal{R}}\xspace}
\newcommand{\Fpsi}[1]{\ensuremath{{\mathcal{F}^{#1}_{\textsf{psi}}}}\xspace}
%%%%%%%%%%%%%%%%%%%%

\newcommand{\etal}{{\sl et~al.}\xspace}
\newcommand{\eg}{{\sl e.g.}\xspace}
\newcommand{\ie}{{\sl i.e.}\xspace}
\newcommand{\apriori}{{\sl a~priori\/}\xspace}

%%%%%%%%%%%%%%%%%%%%%%

\newcommand{\rgets}{\overset{\$}\gets}

\newcommand{\command}[1]{\ensuremath{\text{\sc #1}}\xspace}

%%%%%%%%%%%%%%%%%%%%
\newcommand{\subname}[1]{\ensuremath{\textsc{#1}}\xspace}
\newcommand{\mycmd}[1]{\textsc{#1}}

\newcommand{\codebox}[1]{%
    \begin{varwidth}{\linewidth}%
    \upshape%   no slant in definition/theorem statement!
    \begin{tabbing}%
        \quad\=\quad\=\quad\=\quad\=\kill
        #1
    \end{tabbing}%
    \end{varwidth}%
}

\newcommand{\fcodebox}[1]{%
    \framebox{\codebox{#1}}%
}

\newcommand{\comment}[1]{
    \sl\small\color{black!50} \mbox{ // #1 }
}


%%%%%%%%%%%%%%%%%%%%


    \newtheorem{theorem}{Theorem}
    \newtheorem{definition}[theorem]{Definition}
    \newtheorem{claim}[theorem]{Claim}
    \newtheorem{lemma}[theorem]{Lemma}
    \newtheorem{proposition}[theorem]{Proposition}
\newtheorem{corol}[theorem]{Corollary}
\newtheorem{assumption}[theorem]{Assumption}
\newtheorem{obs}[theorem]{Observation}
\newtheorem{conj}[theorem]{Conjecture}

\newenvironment{proofof}[1]{\begin{proof}[Proof of #1.]}{\end{proof}}
\newenvironment{proofsketch}{\begin{proof}[Proof Sketch]}{\end{proof}}


%%%%%%%%%%%%%%%%%%%%%%%%%

\newcommand{\namedref}[2]{\hyperref[#2]{#1~\ref*{#2}}}
%% if you don't like it, use this instead:
%\newcommand{\namedref}[2]{#1~\ref{#2}}
\newcommand{\chapterref}[1]{\namedref{Chapter}{#1}}
\newcommand{\sectionref}[1]{\namedref{Section}{#1}}
\newcommand{\theoremref}[1]{\namedref{Theorem}{#1}}
\newcommand{\propositionref}[1]{\namedref{Proposition}{#1}}
\newcommand{\definitionref}[1]{\namedref{Definition}{#1}}
\newcommand{\corollaryref}[1]{\namedref{Corollary}{#1}}
\newcommand{\obsref}[1]{\namedref{Observation}{#1}}
\newcommand{\lemmaref}[1]{\namedref{Lemma}{#1}}
\newcommand{\claimref}[1]{\namedref{Claim}{#1}}
\newcommand{\figureref}[1]{\namedref{Figure}{#1}}
\newcommand{\tableref}[1]{\namedref{Table}{#1}}
\newcommand{\subfigureref}[2]{\hyperref[#1]{Figure~\ref*{#1}#2}}
\newcommand{\equationref}[1]{\namedref{Equation}{#1}}
\newcommand{\appendixref}[1]{\namedref{Appendix}{#1}}

\definecolor{darkred}{rgb}{0.5, 0, 0}
\definecolor{darkgreen}{rgb}{0, 0.5, 0}
\definecolor{darkblue}{rgb}{0, 0, 0.5}


%%%%%%%%%%%%%%%%%%%%%%%%%%%%%

\newcommand{\todo}[1]{%
    \mbox{}% prevent marginpar from being on previous paragraph
    \marginpar{%
        \colorbox{red!80!black}{\textcolor{white}{to-do}}%
        \vspace*{-22pt}% hack!
    }%
    \textcolor{red}{#1}%
}

\newcommand{\myparagraph}[1]{\smallskip\noindent{\bf #1:}}
\newcommand{\DB}{\ensuremath{DB}\xspace}
\newcommand{\encode}[1]{\ensuremath{\llbracket #1 \rrbracket}}
\newcommand{\strLen}{e}
\newcommand{\f}[1]{\ensuremath{\mathcal{F}_{\textsc{#1}}}}
\newcommand{\proto}[1]{\ensuremath{\Pi_{\textsc{#1}}}}
\newcommand{\numbins}{\ensuremath{\beta}\xspace}
\newcommand{\cuckoobins}{\ensuremath{m}\xspace}
\newcommand{\binsize}{\ensuremath{\mu}\xspace}
\newcommand{\symsec}{\ensuremath{\kappa}\xspace}
\newcommand{\statsec}{\ensuremath{\lambda}\xspace}
\newcommand{\clientsetsize}{\ensuremath{n}\xspace}
\newcommand{\serversetsize}{\ensuremath{N}\xspace}
\newcommand{\elsize}{\ensuremath{\rho}\xspace}
\newcommand{\binScale}{\ensuremath{c}\xspace}
\newcommand{\pirblocksize}{\ensuremath{b}\xspace}
\newcommand{\cuckooexpansion}{\ensuremath{e}\xspace}
\newcommand{\cuckootable}{\ensuremath{CT}\xspace}

\title{FaSe DB: Fast and Secure Database Joins on Secret Shared Data}
	\author{Payman Mohassel \and Peter Rindal \and Mike Rosulek }

\begin{document}
\maketitle

\begin{abstract}
We present a scalable database join protocol for secret shared data in the honest majority setting. The key features of our protocol is a rich set of SQL like join/select queries and the ability of compose join operations together due to the inputs and outputs being generically secret shared between the parties. Given that the keys being joined on are unique, no information is revealed to any party during the protocol. In particular, not even the size of the join is revealed. All of our protocols are constant round and achieve $O(n)$ communication and computation overhead for joining two tables of $n$ rows. 

In addition to performing database joins our protocol, we implement two applications on top of our framework. The first performs joins between different governmental agencies to identify voter registration errors in a privacy preserving manner. The second application considers the scenario where several organizations wish to compare network security logs to more accurately identify common security threats, e.g. the ip addresses of a bot net. In both cases the practicality of these applications depends on efficiently performing joins on millions of secret shared records. For example, our three party protocol can perform a join on two sets of 1 million records in 4.9 seconds or, alternatively, compute the cardinality of this join in just 3.1 seconds. 
\end{abstract}



\section{Introduction}

We consider the problem of performing SQL-style join operations on secret shared tables with three parties and an honest majority. In particular, the proposed protocol takes two or more arbitrarily secret shared database tables and constructs another secret shared table containing a join of the two tables, without revealing \emph{any} information beyond the secret shares themselves. Our protocol is constant round and  has $O(n)$ computation and communication  overhead to join two tables with $n$ records. Simulation-based security is achieved in the semi-honest setting with an honest-majority. Our protocol can perform inner, left and full joins along with union and arbitrary circuit computation on the resulting table. A central requirement in achieving high efficiency and no leakage is that the join-keys must be unique. 

New techniques \cite{usenix:PSZ14,USENIX:PSSZ15,PSZ16,CCS:KKRT16,PSWW18,CLR17,CHLR18,cryptoeprint:2017:738,RA17,KLSAP17,OOS17,KMPRT17} for performing set intersection, inner join and related functionalities have shown great promise for practical deployment. To name a few, Ion et al. at Google recently deployed a private set intersection sum protocol\cite{cryptoeprint:2017:738} to allow customers of Google Adwords to correlate the impact online advertising has on generating offline sales while preserving user privacy. Pinkas et al. \cite{PSWW18} also introduced a practical protocol that can compute any (symmetric) function of the intersection and associated data. In both cases these protocols can be framed in terms of SQL queries where an inner join is computed followed by an aggregation on the resulting table, e.g. summing a column.  

The majority of these protocols consider the two party setting and are based on various cryptographic primitives, e.g. exponentiation~\cite{cryptoeprint:2017:738}, oblivious transfer\cite{PSWW18}, or fully homomorphic encryption\cite{CLR17}. However, in this work we alter the security model to consider three parties with an honest majority. The motivation is that typical protocols in this setting (e.g.\cite{highthroughput}) require less computation and communication than similar two party protocols, by a factor of at least the security parameter $\kappa=128$. Moreover, we will see that the honest majority enables various algorithms which are orders of magnitude more efficient, e.g. oblivious permutations. 

Given this observation we investigate how to leverage the efficiency gains in the three party setting to construct practical protocols for performing set intersection and other SQL like operations where both the inputs and outputs are secret shared. One critical aspect of this input/output requirement is that join operations can then be \emph{composed} together, where the output of a join can be the input to another. Allowing this composability greatly increases the ability to perform highly complex queries and enables external parties to contribute data simply by secret sharing it between the primary parties which participate in the protocol.

\subsection{Functionality}

Our protocol offers a wide variety of functionality including set intersection, set union, set difference and a variety of SQL-like joins with complex boolean queries. Generally speaking, our protocol works on tables of secret shared data which are functionally similar to SQL tables. This is contrasted by traditional PSI and PSU protocols\cite{usenix:PSZ14,USENIX:PSSZ15,PSZ16,CCS:KKRT16} in that each record is now a tuple of values as opposed to a single key. 

We define our database tables in the natural way. Each table can be viewed as a collection of rows or as a vector of columns. For a table $X$, we denote the $i$th row as $X[i]$ and the $j$th column as $X_j$. 
\iffullversion
Note that each column of a table has the same length but can contain different data types, e.g. $X_1$ is a column of 32 bit fixed point decimal values and $X_2$ is a column of 1024 bit strings.

\fi
Our core protocol requires each table to contain unique values in the column defining the join  (i.e., we can only join on ``unique primary keys''). For example, if we consider the following SQL styled join/intersection query
$$
\texttt{select } X_2 \texttt{ from } X \texttt{ inner join } Y \texttt{ on } X_1 = Y_1
$$
then the join-keys are $X_1$ and $Y_1$. This uniqueness condition can be extended to the setting where multiple columns are being compared for equality. 
\iffullversion
In this case the tuple of these columns must be unique. 
\fi
Later on we will discuss the case when such a uniqueness property does not hold. Our protocols also support a \texttt{where} clause that filters the selection using an arbitrary predicate of the $X$ and $Y$ rows. Furthermore, the \texttt{select} clause can also return a function of the two rows. For example,
\begin{align*}
\texttt{select }&  X_1,max(X_2, Y_2)  \texttt{ from } X \texttt{ inner join } Y \\
\texttt{on }& X_1 = Y_1 \texttt{ where } Y_2 > 23.3
\end{align*}
In general, the supported join operations can be characterized in three parts: 1) The select function $S(\cdot)$ that defines how the rows of $X,Y$ are used to construct each output row, e.g. $S(X,Y)=(X_1, max(X_2,Y_2))$
\iffullversion
in the example above
\fi. 2) The predicate $P(\cdot)$ that defines the \texttt{where} clause, and 3) which columns are being joined on.
\iffullversion
 We require that both $S$ and $P$ must be expressible in the framework of \cite{aby3} which we provide a custom implementation of.
\fi

\iffullversion
So far we have discussed inner join (intersection) between two tables. 
\fi
Several other types of joins are also supported including left and right joins, set union and set minus (difference) and full joins. A left join takes the inner join and includes all of the missing records from the left table. For the records solely from the left table, the resulting table contains \texttt{NULL} for the columns from the right table. Right join is defined symmetrically. A full join is a natural extension where all the missing rows from $X$ and $Y$ are added to the output table.

We define the union of two tables to contain all records from the left table, along with all the records from the right table which are not in the intersection with respect to the join-keys. Note that this definition is not strictly symmetric with respect to the left and right tables due to rows in the intersection coming from the left table. %\footnote{We define union in this way to resolve the ambiguity around which table the records in the intersection should come from. For example, consider two tables each with one row $X[0]=(k, x_2,x_3,...), Y[0]=(k, y_2,y_3,...)$. The union of $X,Y$ with $X_1,Y_1$ being the join-keys should only contain one row $(k,...)$ but which values $x_2,x_3,...$ or $y_2,y_3,...$ that should be in the output table is ambiguous. We therefore define it to be $(k,x_2,x_3,...)$. A full join allows both sets of values to be in the output.}. 
Table minus is similarly defined as all of the left table whose join-column value is not present in the right table. 

Beyond these various join operations, our framework supports two broad classes of operations which are a function of a single table. The first is a general SQL select statement which can perform computation on each row (e.g. compute the max of two columns) and filter the results using a \texttt{where} clause predicate. The second class is referred as an aggregation which perform an operation across all of the rows of a table. For example, computing the sum, counts or the max of a given column. 
\subsection{Our Results}

We present the first practical secure multi-party computation protocol for performing SQL styled database joins with linear overhead and constant round. Our protocol is fully composable in that the input and output tables are generically secret shared between the parties. 
\iffullversion
	No partial information is revealed at any point. 
\fi
	We achieve this result by combining various techniques from private set intersection and secure computation more 
\iffullversion 
	broadly while requiring the joined on keys be unique.
\else
	broadly.
\fi
 We build on the the binary secret sharing technique of \cite{highthroughput} with enhancements described by \cite{aby3}. We then combine this secret sharing scheme with cuckoo hashing\cite{usenix:PSZ14}, an MPC friendly PRF\cite{lowmc} and a custom protocol for evaluating an oblivious switching network\cite{MS13}. Using these building blocks our protocol is capable of computing the intersection of two tables of $n=2^{20}$ rows in 4.9 seconds. 
\iffullversion 
	Alternatively, the cardinality of the intersection can be computed in just 3.1 seconds.
\fi
  Beyond these two specific functionalities, our protocol allows arbitrary computation applied to a shared table. Compared to existing three party protocols with similar functionality (composable), our implementation is roughly $1000\times$ faster. When compared with \emph{non-composable} two party protocol, we observe a larger difference ranging from our protocol being $1.25\times$ slower to $4000\times$ faster depending on what functionality is being computed. 


Building on our proposed protocol we demonstrate it's utility by showcasing two potential applications. The first prototype would involve running our protocol between and within the states of the United States to validate the accuracy of the voter registration data in a privacy preserving way. The Pew Charitable Trust\cite{pew} reported 1 in 8 voter registration records in the United States contains a serious error while 1 in 4 eligible citizens remain unregistered. Our privacy preserving protocol identifies when an individual's address is out of date or more seriously if someone is registered to vote in more than one state which could allow them to cast two votes. 
\iffullversion
Additionally, our protocol can help register eligible citizens. Our protocol ensures that only the minimum amount of information is revealed, namely the identities of individuals with serious registration errors.
\fi 
Due to how the data is distributed between different governmental agencies, it will be critical that our protocol allows for composable operations. We implement this application and demonstrate that it is practical to run at a national scale (quarter billion records) and low cost.

The second application that we consider allows multiple organizations to compare computer security incidents and logs to more accurately identify unwanted activities, e.g. a bot net. Several companies already offer this service including Facebook's ThreatExchange\cite{threat} and an open source alternative\cite{alt_threat}. One of the primary limitations of these existing solutions is the requirement that each organization send their security logs to a central party, e.g. Facebook. We propose using our protocol to distribute the trust of this central party between three parties such that privacy is guaranteed so long as there is an honest majority.


\section{Related Work}

\begin{itemize}
	\item MikePaymanBrent's 2pc paper.

% Private and Oblivious Set and Multiset Operations
%     https://eprint.iacr.org/2011/464.pdf
\item With respect to functionality the most related work is that of Blanton and Aguiar\cite{BA11} which describes a relatively complete set of protocols for performing intersections, unions, set difference, etc. and the corresponding SQL-like operations. Moreover, these operation are composable in that the inputs and outputs are secret shared between the parties. At the core of their technique is use of a generic MPC protocol and an oblivious sorting algorithm to merges the two sets followed by a linear pass over the sorted data where a relation is performed on adjacent items. This technique has the advantage of being very general and flexible. However, the proposed sorting algorithm has complexity $O(n \log^2 n)$ and is not constant round\footnote{It is not constant round when the underlying MPC protocol is not constant round.}. As a result, the implementation from 2011 performed poorly by current standards, intersecting $2^{10}$ items in 12 seconds. The modern protocol \cite{kkrt} which is \emph{not composable} can perform intersections of $2^{20}$ items in 4 seconds. While this difference of three orders of magnitude would narrow if reimplemented using modern techniques, the gap would likely remain large.

% Private Set Intersection:Are Garbled Circuits Better than Custom Protocols? 
%     https://ssltest.cs.umd.edu/~jkatz/papers/psi.pdf
\item  Huang, Evans and Katz\cite{HEK12} also described a set intersection protocol based on sorting. Unlike \cite{BA11}, this work considers the two party setting where each party holds a set in the clear. This requirement prevents the protocol from being composable but allows the complexity to be reduced to $O(n\log n)$. The key idea is that each party locally sorts their set followed by merging the sets within MPC. The protocol can then perform a single pass over the sorted data to construct the intersection. 

% (KS06) Privacy-Preserving Set Operations
%     https://www.cs.cmu.edu/~leak/papers/set-tech-full.pdf
% (MF06) Efficient Polynomial Operations in the Shared-Coefficients Setting 
%     https://pdfs.semanticscholar.org/80ca/9f56cffce534e047d049884736ff16204958.pdf
\item Another line of work was begun by Kissner and Song\cite{KS06} and improved on by \cite{MF06}. Their approach is based on the observation that set intersection and multi-set union have correspondence to operation on polynomials. A set $S$ can be encoded as the polynomial $\hat S(x)= \prod_{s\in S}(x-s)\in \mathbb{F}[x]$. That is, the polynomial $\hat S(x)$ has a root for all $s\in S$. Given two such polynomials, $\hat S(x), \hat T(x)$, the polynomial encoding the intersection is $\hat S(x)+\hat T(x)$ with overwhelming probability given a sufficiently large field $\mathbb{F}$. Multi-set union can similarly be performed by multiplying the two polynomials together. Unlike with normal union, if an item $y$ is contained in $S$ and $T$ then $\hat S(x)\hat T(x)$ will contain two roots at $y$ which is often not the desired functionality. This general idea can be transformed into a secure multi-party protocol using oblivious polynomial evaluation\cite{???} along with randomizing the result polynomial. The original computational overhead was $O(n^2)$ which can be reduced to the cost of polynomial interpolation $O(n\log n)$ using techniques from \cite{MF06}. The communication complexity is linear. In addition, this scheme assumes an ideal functionality to generate a shared Paillier key pair. We are unaware of any efficient protocol to realize this functionality except for \cite{RSA:HMRT12} in the two party setting.

This general approach is also composable. However, due to randomization that is performed the degree of the polynomial after each operation doubles. This limits the practical ability of the protocol to compose more than a few operations. Moreover, it is not clear how this protocol can be extended to support SQL-like queries where elements are key-value tuples.

% C. Hazay and K. Nissim. Efficient set operations in the presence of malicious adversaries. In PKC, 2010.
%   http://citeseerx.ist.psu.edu/viewdoc/download?doi=10.1.1.454.1521&rep=rep1&type=pdf
\item Hazay and Nissim introduce a pair of protocols computing set intersection and union which are also based on oblivious polynomial evaluation where the roots of the polynomial encode a set. However, these protocols are restricted to the two party case and is not composable. The non-composability comes from that fact that only party constructs a polynomial $\hat S(x)$ encoding their set $S$ while the other party obliviously evaluates it on each element in their set. The result of these evaluations are compared with zero\footnote{The real protocol is slightly more complicated than this.}. These protocols have linear overhead and can achieve security in the malicious setting.

\item (PSU) Keith Frikken 2007. Privacy-preserving set union. 

\end{itemize}


\section{Preliminaries} \label{sec:prelim}


\subsection{Security Model}

Our protocols are presented in the semi-honest three party setting with an honest majority. That is, our protocols are computationally secure conditioned on the adversary corrupting at most one of the three parties. 



Throughout the exposition we will assume these three ``server" parties provide the sets which are computed on. However, in the general case the sets being computed on can be privately input by an arbitrary party which does not participates in the computation. In particular, these ``client" parties will secret share there set between the server parties and reconstruct the output shares that are indented for them. This setting is often referred to as the client server model\cite{aby3, secureML}. Later on we will also consider a setting with five parties that can tolerate the adversary corrupting any two of them. However, we will explicitly state when this alternative model is being considered.



\subsection{Notation}

We will use the notation that $\share{x}$ is an additive secret sharing of the value $x$.

permutations: define that a permutation can be applied to an integer $i$ which maps the output position $i$ to the input position $\pi(i)$. Or, a permutation can be applied to a vector $V$ which applied the permutation to $V$, i.e. $\pi(V)=\{V_{\pi(1)}, ..., V_{\pi(n)}\}$. 


define a function image and preimage.


\subsection{MPC Protocol}

State that we use the ABY 3 framework. 

\subsection{Cuckoo Hash Tables}

The core data structure that our protocols employ is a cuckoo hash table. This data structure is parameterized by a capacity $n$, two (or more) hash functions $h_0, h_1$ and a vector $T$ which has $m=O(n)$ slots, $T_0, ..., T_m$. For any $x$ that has been added to the hash table, there is an invariant that $x$ will be located at $T_{h_0(x)}$ or $T_{h_1(x)}$. Testing if an $x$ is in the hash table therefore only requires inspecting these two locations. $x$ is added to the hash table by inserting $x$ into slot $T_{h_i(x)}$ where $i\in \{0,1\}$ is picked at random. If there is an existing item at this slot, the old item $y$ is removed and reinserted at its other hash function location. Given a hash table with $m=1.5n$ slots and three hash functions, then $n$ items can be inserted with overwhelming probability\cite{DRRT18}. 
\section{Our Construction}\label{sec:construction}



\subsection{Overview}

Our core protocol is a technique for obliviously mapping together records that have equal keys. In particular, for each secret shared row $\share{X[i]}$ our protocol obliviously maps the row $\share{Y[j]}$ to the $i$th position of a new table $\share{Y'}$ if the \emph{joined on} columns/keys are equal. In the event that no such $Y[j]$ exists then an arbitrary $j$ is used instead. Once the mapping is performed the output table can be constructed by an MPC protocol\cite{aby3} that compares the $i$th row of $\share X$ and $\share{Y'}$. 

Without loss of generality let us assume that the columns $X_0$ and $Y_0$ are being joined on. Our protocol begins by generating a \emph{randomized encoding} for each of the secret shared elements $\share x\in \share{X_0}$ and $ \share y\in \share{Y_0}$. \figureref{fig:randomized-encode-ideal} contains the ideal functionality for this encoding which take secret shares from the parties, apply a PRF $F_k$ to the reconstructed value using a internally sampled key $k$, and return the resulting value to one of the three parties. For $\share x\in \share{X_0}$, party 0 will learn $F_k(x)$ while party 1 will learn $F_k(y)$ for $\share y\in \share{Y_0}$.

Party 1 proceeds by constructing a \emph{secret shared} cuckoo hash table $\share{\hat Y}$ for the rows of $\share{Y}$ where the hash function values employed are defined as $h_i(y) = H( i || F_k(y))$. That is, party 1 knows the hash function values but the contents of the hash table remains secret shared between the parties. To prevent the parties 0 and 2 from learning information about the randomized encodings $F_k(y)$, party 1 must obliviously permute their shares to the desired position of the cuckoo hash table. We achieve this using a three party oblivious permutation protocol which further randomizes the secret shares.

It is now the case that $\share{\hat Y}$ is a valid cuckoo hash table of $\share Y$ which is in a secret shared format. Party 0, who knows the randomized encodings $F_k(x)$ for all $\share x\in \share{X_0}$, now must query $\share{\hat Y}$ at the slots indexed by $h_i(x)= H( i || F_k(x))$ and compare this with the corresponding row of $\share X$. In particular, assuming we use two hash function, then party 0 constructs an \emph{oblivious switching network} that maps the shares $\share{\hat Y[{h_0(x)}]}$ and $\share{\hat Y[{h_1(x)}]}$ to be aligned with $\share x$.

Once the shares  $\share{\hat Y[{h_0(x)}]}, \share{\hat Y[{h_1(x)}]}$ and $\share x$ are aligned, the parties employ an MPC protocol to compare the joined on columns to compute a secret shared bit denoting whether $x$ matches with one of the rows. When computing an inner join query, only rows of $X$ where the comparison bit is set to one are considered valid while the other rows are set to \texttt{NULL}. Note that when columns of $Y$ are selected the corresponding values are obtained from $\share{\hat Y[{h_i(x)}]}$ for the $i$ that the comparison succeeded on. Left joins work in a similar way except that all rows of $X$ are included while the comparison bit is used to select the columns of $Y$ or set the fields to \texttt{NULL}. Finally, unions can be computed by including all of $Y$ in the output and all of the rows of $X$ where the comparison bit is set to zero. Regardless of the type of join, the protocols do not reveal any information about the set. In particular, not even the cardinality of the join is revealed.

\subsection{Randomized Encodings}

Randomized encodings enable the parties to coordinator their secret shares without revealing the underlying values. Crucially, the equality  of two randomized encodings implies the equality of the encoded value but nothing more. The ideal functionality of the encoding process is given in \figureref{fig:randomized-encode-ideal}. It considers two commands which allow the parties to initialize the internally stored key $k$ and later generate encoding under that key. In particular, the parties are allowed to send secret shares of a value $x$  to the ideal functionality and destinate which party should learn the encoding. This functionality has several interesting properties. First, all encoding that have not been observed are uniformly distributed in that parties view. Secondly, learning an encoding $F_k(x)$ does not reveal any information about $x$ beyond being about to compare it for equality with other encodings. 

\begin{figure}[ht]
	\framebox{\begin{minipage}{0.95\linewidth}
			Parameters: $N$ parties denoted as party 0 through party N-1. The input domain $\{0,1\}^\sigma$ and output domain $\{0,1\}^\ell$ for a pseudorandom function $F$.
			
			\begin{enumerate}
				\item[] [Key Gen] Upon receiving command $(\textsc{KeyGen})$ from all parties, sample a uniformly random key $k$ from the key space of $F$ and store it internally.
				
				\item[] [Encode] Upon receiving command $(\textsc{Encode}, \share x, i)$ from all parties, if $k$ is uninitialized compute $F_k(x)$ and send it to party $i$. 
			\end{enumerate}
	\end{minipage}}
	\caption{The Randomized Encoding ideal functionality \f{encode}}
	\label{fig:randomized-encode-ideal}	
\end{figure}


\paragraph{LowMC Encodings.}
We consider two protocol for realizing \f{encode}. The first is optimized with respect to computational overhead and is based the LowMC circuit\cite{lowmc} which computes a block cipher. With this approach we use the framework of \cite{aby3, highthroughput} to evaluate the a circuit implementing this PRF and reveal the output to the designated party. When coupled with the honest majority MPC protocols\cite{aby3, highthroughput}, this approach results in extremely high throughput. For instance, our implementation can compute one million encodings in a few seconds.

LowMC is a family of MPC optimized block ciphers based on a binary substitution-permutation network. The cipher is parameterized by a block size $n$, keys size $\kappa$, s-boxes per layer $m$ and the desired data complexity $d$. Given the desired security level, e.g. 128 bits, the required number of rounds $r$ can then be computed as a function of these parameters. The structure of the cipher is specifically optimized to reduced the number of s-boxes (\textsc{and} gate) and the number of rounds. For each of the $r$ rounds, the cipher adds part of the key to the current state, multiplies it with a public binary matrix and then applies 3 bit s-boxes in parallel to a subset of the state. With respect to performance metrics, the most costly operation is the application of the s-boxes and the number of rounds required. 

An important observation of our protocol is that the adversary only sees a bounded amount of block cipher output. In particular, the number of blocks observed $d$ is exactly the size of the table which is being encoded. For our implementation we set $d=2^{30}$ and optimize the remaining parameters. Our second observation is that our protocol uses the cipher as a PRF and does not require a excessive number of output bits. The desired property of the encodings is that the probability of spurious collisions between encodings is negligible in the statistical security parameter $\lambda$. Given table sizes of $|X|$ and $|Y|$, this can be ensured by bounding $n$ such that $n-\log |X|-\log |Y|\geq \lambda$. Considering the standard of setting $\lambda=40$, we observe that $n=80$ gives a sufficient margin for realistic table sizes. The remaining parameter $m$ was optimized empirically and set to be $m=14$ which resulting in $r=13$. This results in the evaluation of the LowMC requiring 13 rounds of communication and a total of 546 \textsc{and} gates (bits of communication).



\paragraph{Diffie-Hellman Encodings.} Our second approach uses a Diffie-Hellman styled assumption to construct a PRF on shared input. In particular, for the share $\share x$ the protocol computes $F_k(x) = g^{xk}$. This approach has the advantage of not computing a PRF circuit within MPC which can be costly in some settings.


\todo{talk about how to convert the shares.  We have binary shares and we need mod $p$ shares... }


%\paragraph{Encodings Long Elements.}\todo{If the element is too large to be encoded above, we can choose a random binary matrix and (locally) multiply the shares with it. We then encode. Given that this matrix is chosen after the shares, everything should be good.}

\subsection{Oblivious Switching Network}

A switching network was introduced by Mohassel and Sadeghian\cite{MS13} as a circuit that can obliviously transform a vector $A=\{A_1,...,A_n\}$ such that the output is $A'=\{A_{\pi(1)}, ..., A_{\pi(m)}\}$ for an arbitrary function $\pi : [m]\rightarrow[n]$. The protocol of \cite{MS13} was designed in the two party setting where the first party inputs $A$ while the second party inputs a description of $\pi$. More recently Carmer et al. \cite{CMRS18} generalized the construction and demonstrated how this technique can be utilized for private set intersection. In both cases these switching networks require  $O(n\log n)$ cryptographic operations. Building on this general paradigm, we introduce a new oblivious switching network protocol tailored for the honest majority setting which significantly improves the efficiency. In particular, our protocol has linear overhead and is information theoretic. 

The ideal functionality of our protocol is given in \figureref{fig:perm-ideal}. This functionality considers three parties, a \emph{programer}, a \emph{sender} and a \emph{receiver}. The programmer has a description of an arbitrary function  $\pi:[m]\rightarrow[n]$ while the sender a vector $A$ containing $n$ elements each consisting of $\sigma$ bits. At the completion of the protocol, the programmer and the receiver should hold a 2-out-of-2 secret sharing of $A'=\{A_{\pi(1)}, ..., A_{\pi(m)}\}$.


The protocols and functionality described below assume the vector being transformed is the private input of the sender. However, our larger join protocols require the transformations to be applied to secret shared vectors. In particular, parties 0 and 1 both hold secret shares of $A$. This is achieved by using the oblivious switching protocol to transform the shares of the sender. The programmer who knows the program $\pi$ can then locally permute their local share and combine this with the output of the oblivious switching protocol.

\begin{figure}[ht]
	\framebox{\begin{minipage}{0.95\linewidth}
			Parameters: $3$ parties denoted as the \emph{programmer}, \emph{sender} and \emph{receiver}. Elements are strings in $\{0,1\}^\sigma$. An input vector size of $n$ and output size of $m$.
			
			\begin{enumerate}
				\item[] [Switch] Upon the command $(\textsc{switch}, \pi)$ from the \emph{programmer} and $(\textsc{switch}, A)$ from the \emph{sender}, the functionality performs:
				\begin{enumerate}
					\item Interpret $\pi: [m]\rightarrow [n]$ as a function and $A\in \{0,1\}^{n\times \sigma}$ as a vector of $n$ elements. 
					\item Uniformly sample two $m$ elements vector $B^0, B^1\gets \{0,1\}^{m\times \sigma}$ such that for all $i\in [m], A_{\pi(i)} = B^0_i \oplus B^1_i$.
					\item Send $B^0$ to the \emph{programmer} and $B^1$ to the \emph{receiver}.
				\end{enumerate}
			\end{enumerate}
	\end{minipage}}
	\caption{The Oblivious Switching Network ideal functionality \f{switch}}
	\label{fig:perm-ideal}	
\end{figure}


\paragraph{Permutation Network}

We begin with a restricted class of switching networks where the programming function $\pi$ is injective. These types of programs do not allow a single input element $A_i$ to be mapped to more than one location in the output vector. 

Our protocol begins by having the programmer sampling a two random functions $\pi_0,\pi_1$ such that $\pi_1 \circ \pi_0 = \pi$, $\pi_0$ is bijective and $\pi_1$ is injective. The programmer sends $\pi_0$ and a uniform vector $S$ of $n$ elements to the sender who sends $T := \{A_{\pi(1)} \oplus S_0, ...,A_{\pi(n)} \oplus S_n \}$ and  $\pi_1$ to the receiver. The final shares of the permuted $V$ are defined as the programmer holding $\{S_{\pi_1(1)}, ..., S_{\pi_1(m)}\}$ and the receiver holding $\{T_{\pi_1(1)}, ..., T_{\pi_1(m)}\}$.



%LWZ11 = https://eprint.iacr.org/2011/429.pdf
The simulatability of this protocol is straight forward and follows a similar logic as the oblivious shuffle protocol of Laur et al.\cite{LWZ11}. The view of the sender contains a uniformly distributed permutation of $n$ elements $\pi_0$ and a uniformly distributed vector $S$. Similarly, the view of the receiver contains $\pi_1: [m]\rightarrow [n]$ which is uniformly distributed (when $\pi_0$ is unobserved) and the vector $T$ is uniformly distributed given that it is masked by $S$. One important observation of this simulation is that $\pi_0,S$ can be generated locally by parties 0 and 1 using a common source of randomness, e.g. a seeded PRG. This reduces the rounds to 1 and linear communication complexity. 

\paragraph{Universal Switching Network}\label{sec:switch}

A universal switching network with an \emph{arbitrary} program $\pi : [m]\rightarrow [n]$ can be constructed in three phases\cite{MS13, CMRS18}. In particular, the input vector $A$ will have three transformation applied $A\overset{\pi_1}{\rightarrow}B\overset{\pi_2}{\rightarrow}C\overset{\pi_3}{\rightarrow}D=\pi(A)$.
\begin{enumerate}
	\item $B:=\pi_1(A)$:  The input vector $A$ is permuted by the injective function $\pi_1:[m]\rightarrow[n]$ such that if $\pi$ maps an input position $i$ to $k$ outputs positions (i.e. $k=|preimage(\pi,i)|=|\{ j : \pi(j)=i \}|$), then there exists a $j$ such that $\pi_1(j)=i$  and $\{\pi_1(j)+ 1,...,\pi_1(j )+k \} \cap image(\pi) = \emptyset$. That is, wherever position $i$ is mapped by $\pi_1$, it should be followed by $k-1$ input that do not appear in the final output. The parties then use a permutation network to compute $B:=\pi_1(A)$.
	
	\item $C:=\pi_2(B)$: The intermediate vector $B$ is transformed by a duplication network $\pi_2:[m]\rightarrow[m]$ such that if position $A_i$ is mapped to $k$ positions in $\pi(A)$, then $\{ C_{j},...,C_{j+k}\} = \{A_i\}$ where $\pi_1(j)=i$. That is, $C$ takes $B$ and copies $B_{j}$ into the next $k-1$ positions. 
	
	\item $D:=\pi_3(C)$: The final transformation $\pi_3:[m]\rightarrow[m]$  permutes $C$ to have the same ordering as $\pi(A)$. That is, the elements $\{ C_{j},...,C_{j+k}\}$ which all have the value  $A_i$ are arbitrary mapped to the $k$ positions $\{ j : \pi(j)=i \}$.
\end{enumerate}
Observer that steps 1 and 3 can both be implemented using the oblivious permutation protocol.% However, note that our oblivious permutation functionality is defined for $m\leq n$ while the switching network has no such restriction. This can be overcome by artificially padding the input vector $V$ with dummy items to be of size $\max(m, n)$. 

What remains is how to efficiently implement the duplication network $\pi_2:[m]\rightarrow[m]$. This transformation can be characterized by a bit vector $b$ of length $m-1$ where the $i$th bit denotes whether the item at position $i$ should have the same value as position $i+1$. This observation gives rise to a natural protocol: for $i\in [m-1]$, if $b_i=1$ then use MPC to copy $B_i$ into $B_{i+1}$. The primary challenge is to achieve this while using a constant number of communication rounds which prevents the use of a generic MPC protocol such as \cite{aby3, highthroughput}.

To obliviously select $B_i$ or $B_{i+1}$ conditioned on the programming bit $b_i$ we require the resulting value $C_{i+1}$ be secret shared. In particular, we consider the setting there the programmer knowns the programming bit $b_i$ while $B_i$ and $B_{i+1}$ are private input of the sender. At the end the programmer and sender will hold secret share $C^0_{i+1}, C_{i+1}^1$ of $C_{i+1}:=b_i B_i + (1-b_1)B_{i+1}$. The sender begins by sampling three random strings $C_{i+1}^1, w_0,w_1\gets \{0,1\}^\sigma$ and a random bit $\phi\gets \{0,1\}$. They construct two messages $m_0=B_{i+1}^1\oplus C_{i+1}^1\oplus w_\phi$ and $m_1= B_i\oplus C_{i+1}^1 \oplus w_{\phi\oplus 1}$. The sender sends $w_0,w_1$ to the receiver and sends $m_0,m_1,\phi$ to the programmer who sends $\rho=\phi\oplus b_i$ to the receiver. The final share are constructed by having the receiver send $w_\rho$ to the programmer who computes $C_{i+1}^0:=m_{b_i}\oplus w_{\rho}$.

The simulation of this protocol has two key parts. First, party 0 learns only one of the keys $w_0,w_1$ which determines which of the secret share $(B_{i+1}\oplus C_{i+1}^1$ or $  B_i\oplus C_{i+1}^1)$ they can one-time-pad decrypt. As such, the other share is uniformly distributed in their view. Similarly, the bit $\rho$ that the programmer sends to the receiver is uniformly distributed given that $\phi$ is not contained in the view of the receiver . The remaining messages $w_0,w_1,\phi$ which are uniformly sampled are trivial to simulate. Moreover, sending these messages can be optimized away when a PRG seed is pre-share between the appropriate parties.

The protocol just described considers the setting where the messages $B_i,B_{i+1}$ are the private input of the receiver. However, we require that at each iteration the messages being selected is either $C_i$ or $B_{i+1}$ where the former was computed in the previous iteration and is secret shared between the programmer and sender. Fortunately, a trivial modification yields the desired functionality. The sender simply utilizes their share of $C_i$ instead if $B_i$ while the programmer can now compute $C_{i+1}^0:=m_{b_i}\oplus w_{\rho}\oplus b_iC_{i}^0$. It is now the case that both shares of $C_i$ are obliviously multiplied by $b_i$. \figureref{fig:switching-net} provides a formal description of the full switching network protocol.

\begin{figure}[ht!]
	\framebox{\begin{minipage}{0.95\linewidth}\small
			Parameters: $3$ parties denoted as \emph{programmer}, \emph{sender} and \emph{receiver}. Elements are strings in $\{0,1\}^\sigma$. An input vector size of $n$ and output size of $m$.
			
			\begin{enumerate}
				\item[] [Permute] Upon the command $(\textsc{Permute}, \pi)$ from the \emph{programmer} and $(\textsc{Permute}, A)$ from the \emph{sender}.  $\pi: [m]\rightarrow [n]$ is parsed as a \emph{injective} function and  $A\in \{0,1\}^{n\times \sigma}$ as a vector of $n$ elements. Then:
				\begin{enumerate}
					\item The \emph{programmer} samples a uniformly random bijective function $\pi_0 : [n]\rightarrow[n]$ and computes the injective function $\pi_1 :[n] \rightarrow[m]$ such that $\pi_1\circ \pi_0 = \pi$.  $\pi_0 $ and a random vector $S\gets \{0,1\}^{n\times \sigma}$ are sent to the \emph{sender}.
					\item The \emph{sender} computes and sends $B := \{ A_{\pi_0(1)} \oplus S_1, ..., A_{\pi_0(n)} \oplus S_n\}$ to the \emph{receiver}.
					\item The \emph{programmer} sends $\pi_1$ and a random vector $T\gets\{0,1\}^{m\times\sigma}$ to the \emph{receiver} who outputs $C^0:=\{B_{\pi_1(1)} \oplus T_1,...,B_{\pi_1(m)}\oplus T_m\}$. The \emph{programmer} outputs $C^1:=\{ S_{\pi_1(1)}\oplus T_1,...,S_{\pi_1(m)}\oplus T_m\}$.
				\end{enumerate}
				
				\item[] [Switch] Upon the command $(\textsc{Switch}, \pi)$ from the \emph{programmer} and $(\textsc{Switch}, A)$ from the \emph{sender}. $\pi: [m]\rightarrow [n]$ is parsed as a function and  $A\in \{0,1\}^{n\times \sigma}$ as a vector of $n$ elements. Then:
				\begin{enumerate}
					\item If $n<m$, the \emph{sender} redefines $A$ to be $A := A || \{0\}^{(m-n)\times \sigma}$ and all parties redefine $n:=m$.
					\item The \emph{programmer} samples an injective function $\pi_1:[m]\rightarrow [n]$ such that for $i\in image(\pi)$ and $k=|preimage(\pi, i)|$, there exists a $j$ where $\pi_1(j)=i$ and $\{\pi_1(j+1), ...,\pi_1(j+k) \}\cap image(\pi)=\emptyset$.
					
					The \emph{programmer}  sends $(\textsc{Permute}, \pi_1)$ to \proto{switch} and the \emph{sender} sends $(\textsc{Permute}, A)$. The \emph{programmer} receives $B^{0}\in \{0,1\}^{m\times \sigma}$ in response and the \emph{receiver} receives $B^{1}\in \{0,1\}^{m\times \sigma}$. 
					
					\item The \emph{programmer}  computes the vector $b\in\{0,1\}^{m}$ such that for $i\in image(\pi)$ and $k=|preimage(\pi, i)|$, $b_j = 0$ and $b_{j+1}=...=b_{j+k}=1$ where $\pi_1(j)=i$.
					
					The \emph{receiver} samples three $m$ element vectors $C^{1}, W^0,W^1\gets \{0,1\}^{m\times \sigma}$ and $\phi\gets\{0,1\}^m$. They set $C^{1}_1:=B^{1}_1$ and computes 
					\begin{align*}
						M^0_i&:= B^1_{i}\ \ \, \oplus C^{1}_i \oplus W^{\phi_i}_i\\
						M^1_i&:= C^1_{i-1} \oplus C^{1}_i \oplus W^{\phi_i\oplus 1}_i
					\end{align*}
					for $i\in \{2,...,m\}$. The \emph{receiver} sends $M,\phi$ to the \emph{programmer} and $C^{1},W$ to the \emph{sender}. The \emph{programmer} sends $\rho:=\phi\oplus b$ to the \emph{sender} who responds with $\{ W^{\rho_i}_i : i\in [m] \}$. The \emph{programmer} defines $C^{0}_1:=B^{0}_1$ and computes 
					$$
						C^{0}_i:= M^{b_i}_i \oplus W^{\rho_i}_i\oplus b_iC^{0}_{i-1}
					$$
					for $i\in \{2,...,m\}$.
					\item The \emph{programmer} computes the permutation $\pi_3$ such that for  $i\in image(\pi)$ and $k=|preimage(\pi, i)|$, $\{\pi_3(\ell) : \ell\in preimage(\pi, i)\}=\{j, ..., j +k\}$ where $i=\pi_1(j)$.	The \emph{programmer} sends $(\textsc{Permute}, \pi_3)$ to \proto{switch} and the \emph{sender} sends $(\textsc{Permute}, C^{1})$.  The \emph{programmer} receives $S\in \{0,1\}^{m\times \sigma }$ in response. The \emph{receiver} receives and outputs $D^{1}\in \{0,1\}^{m\times \sigma }$.
					
					The \emph{programmer} outputs $D^{0}_i:=S_i\oplus C^{0}_{\pi_3(i)}$ for $i\in [m]$.
				\end{enumerate}
			\end{enumerate}
	\end{minipage}}
	\caption{The Oblivious Switching Network protocol \proto{switch}. }
	\label{fig:switching-net}	
\end{figure}


\paragraph{Arithmetic Shares}

The presentation above is framed in terms of binary secret shares. However, the same protocol strategy can work equally well for arithmetic shares. The main modification is that XOR operations in $\mathbb{F}_{2^\sigma}$ need to be appropriately replaced with addition and subtraction operations in $\mathbb{Z}_{2^\sigma}$. 


\subsection{Join Protocols}\label{sec:join}

The full join protocol can be constructed using the presented building blocks. When joining two tables $X,Y$ our protocol can be divided into four phases:

\begin{enumerate}
	\item Compute randomized encodings of the joined on column(s). 
	\item Party 1 constructs a cuckoo table for table $Y$ and arranges the secret shares using an oblivious permutation protocol. 
	\item For each row $x$ in $X$, party 0 uses an oblivious switching network maps the corresponding location $i_1,i_2$ of the cuckoo hash table to a secret shared tuple $(x, y_{i_1}, y_{i_2})$.
	\item The joined on column(s) of $x$ is then compared to that of $y_{i_1}, y_{i_2}$. If there is a match the output values are constructed. Otherwise the output row is set to \texttt{NULL}.
\end{enumerate} 

\paragraph{Randomized Encodings}
We begin by generating randomized encodings of the columns being joined on. For example, 
$$
	\texttt{select }* \texttt{ from } X \texttt{ inner join } Y \texttt{ on } X_1 = Y_1 \texttt{ and } X_2 = Y_3
$$
In this case there are two joined on columns, $X_1,X_2$ from $X$ and $Y_1,Y_3$ from $Y$. Our protocol requires that the parties generate a randomized encoding for each row of $X$ and $Y$. That is, the parties must send a secret share of $\share{X_1[i] || X_2[i]}$ and $\share{Y_1[i] || Y_3[i]}$ to \f{encode} where $||$ denotes concatenation. 

There are two major challenges to efficiently implement this functionality in a composable setting. First, the encoding functionality takes as input a secret shared value consisting of $\sigma$ bits. Moreover, when implemented using the LowMC block cipher optimal performance is achieved with a block size of $\sigma=80$ bits. However, the number of bits being encoded is dependent on the column sizes and which columns are being joined on. Instead of changing the functionality of the encoding procedure we show that a preprocessing step that reduced the size of the inputs without degrading the security properties of the encodings. 

Specifically, once the tables being joins on have been specified, the parties execute a coin flipping protocol to jointly pick a random string $s\in \{0,1\}^\kappa$. Given $s$ the parties deterministically generate a random matrix $E\gets\{0,1\}^{m\times \sigma}$. The parties can then locally compute $\share{E_x}=\share{X_1[i] || X_2[i]} E$ and $\share{E_y}=\share{Y_1[i] || Y_3[i]} E$ where $m$ denotes the total bit length of the columns being joined on. These secret shares can then be forwarded to the \f{encode}

To preserve correctness and security we require the probability of the following game outputting 1 be negligible in $\lambda$. 
\begin{quote}
Have the adversary select a $X\subset \{0,1\}^{m}$ of size $poly(\lambda)$ and then uniformly at random sample $E\gets \{0,1\}^{m\times \sigma}$. If there exists distinct $x_1,x_2\in X$ such that $x_1E = x_2E$, output 1, otherwise 0. 
\end{quote}

\todo{Show that this holds and that $\sigma=80$ is chill for $\lambda = 40$.}

The second challenge we face has to do with the composibility of our protocols. Namely, after a previous join operation, some (or all) of the rows being joined can be invalid. We require that the randomized encoding does not reveal which rows are invalid. This is achieved by secret sharing a bit for each row which encodes whether or not the row is currently valid. For tables that are input by a party, these bits are publicly set to 1. Given this bit \share{b}, the parties can generate a random $\sigma$ bit share $\share{r}$ for each value $\share{x}$ that is encoded. Before $\share{xE}$ is sent to \f{encode}, the parties compute $\share{x'}:=\share{xE}\oplus \share{\overline{b}}\share{r}$ and send $\share{x'}$ instead. In the event that the current row is valid, this alteration has no effect on the computation due to \share{\overline{b}}\share{r} being a secret sharing of zero. However, when the row was invalid the value of $x'$ is uniformly distributed. Conditioned on $x'$ not colliding with another value being encoded, the resulting encoding is uniformly distributed and unique with overwhelming probability. 

\todo{Pr. of collision.}

Putting everything together, the parties will jointly sample a random binary matrix $E\gets\{0,1\}^{m\times \sigma}$ if $m>\sigma$ and $E=I$ otherwise. For each set of join on columns $\share{Z_{i_1}},...\share{Z_{i_l}}$ that are to be encoded, they parties jointly sample a secret shared value $\share{r}$ where $r\in\{0,1\}^\sigma$ and compute $\share{z'}:=\share{Z_{i_1}||...||Z_{i_l}}E \oplus \share{\overline{b}}\share{r}$. \share{z'} is sent to \f{encode} who returns the encoding for that row to the appropriate party. In particular, party 0 learns the randomized encodings $\mathbb{E}_x$ for the joined on columns of the left table $X$ and party 1 learns the encodings $\mathbb{E}_y$ for the right table $Y$.

\paragraph{Constructing the Cuckoo Table}

The next phase of the protocol is for party 1 to construct a secret shared cuckoo table for $Y$ where each row is inserted based on its encoding $\mathbb{E}_y$. Party 1 locally inserts the encodings $\mathbb{E}_y$ into a plain cuckoo hash table $T$ with $m\approx 1.5|\mathbb{E}_y|$ slots using the algorithm specified in \sectionref{sec:prelim}. Party 1 samples an injective function $\pi : m\rightarrow m$ such that for the $i$th $e\in \mathbb{E}_y$ and $T[j]=e$, $\pi(j)=i$. That is, $\pi$ defines the mapping from each row's original position in the table $Y$ to the corresponding position in the cuckoo table $T$.

Recall that parties 0 and 1 respectively hold a 2-out-of-2 secret sharing $Y^0,Y^1$ such that $Y=Y^0\oplus Y^1$. 
Party 1 sends $(\textsc{Switch}, \pi)$ to \f{switch} and party 0 sends $(\textsc{Swich}, Y^0)$\footnote{Note that $Y^0$ has $n$ rows while $\pi$ is defined with an input vector containing $m\approx 1.5n$ rows. Party 0 will pad $Y^0$ with $m-n$ rows which contain all zeros. These will be mapped to the empty slots of the cuckoo table by the permutation/switching network.}. In response \f{switch} sends $\hat Y^{0,1}$ to party 1  and $\hat Y^{0,0}$ to party 2. Party 1 defines their output of this phase as $\hat Y^0:=\hat Y^{0,1} \oplus \pi(Y^1)$ and party 2 defines their output as $\hat Y^{1} =\hat Y^{0,0}$.

It is now the case that $\hat Y = \hat Y^0\oplus \hat Y^1$ is a valid secret shared cuckoo hash table of the original table $Y$. In particular, for a given row $Y[i]$ with encoding $e=\mathbb{E}_y[i]$, there exists a $j\in \{h_1(e),h_2(e), h_3(e)\}$ such that  $\hat Y[j] = Y[i]$. Here, the $h_i$ functions are hash functions used to construct the cuckoo table $T$. Another important observation is that $\pi$ is a permutation and therefore the more efficient permutation protocol can be used in place of the universal switching protocol.

We note that some of the columns of the tables may be secret shared in arithmetic group as opposed to binary shares. In this case the switching network will use the appropriate arithmetic operation as note in \sectionref{sec:switch}. 

\paragraph{Selecting from the Cuckoo Table}

The next phase of the protocol is to select the appropriate rows of $\hat Y$ and compare them with each row of $X$. This is achieved with party 0's knowledge of the randomized encodings $\mathbb{E}_x$. Namely, party 0 knows that if the joined on columns of the $i$th row $X[i]$ will match with a row from $Y$, then this row will be at $\hat Y[j]$ for some $j\in \{h_1(e),h_2(e), h_3(e)\}$ where  $e=\mathbb{E}_x[i]$. 

To obliviously compare these rows, party 0 will construct three switching networks with programming $\pi_1,\pi_2,\pi_3 : n\rightarrow m$ such that if $h_l(\mathbb{E}_x[i])=j$ then $\pi_l(i)=j$. Each of these will be used to construct a secret shared table $\share{\widetilde{Y}^1},\share{\widetilde{Y}^2},\share{\widetilde{Y}^3}$ which are the result of applying the switching networks $\pi_1,\pi_2,\pi_3$ to $\share{\hat Y}$. In particular, for the $i$th row $X[i]$ it is now the case that if $X[i]$ has a matching row in $Y$ then it will be contained at  $\widetilde{Y}^1[i],\widetilde{Y}^2[i]$ or ${\widetilde{Y}^3}[i]$. 

An important edge case to consider is when a collision among $\{h_1(e),h_2(e), h_3(e)\}$ occurs where $e= \mathbb{E}_x[i]$. In this case, two (or all three) of the rows $\share{\widetilde{Y}^1}[i],\share{\widetilde{Y}^2}[i],\share{\widetilde{Y}^3}[i]$ will be the same. As such, when we compare $X[i]$ with these rows there can be two (or three) matches and the comparison circuit will not know which to select. We overcome this by padding the set $\{h_1(e),h_2(e), h_3(e)\}$ to the desired size with arbitrary values from $[m]$. Therefore it is guaranteed that for the $i$th row, the joined on columns of $\{\share{\widetilde{Y}^1}[i],\share{\widetilde{Y}^2}[i],\share{\widetilde{Y}^3}[i]\}$ will  have at most one intersection with $X[i]$.


\paragraph{Inner Join}

Given the four secret shared tables $\share{X},\share{\widetilde{Y}^1},\share{\widetilde{Y}^2},\share{\widetilde{Y}^3}$ as described above, the parties do a learn pass over the $n$ rows to construct the inner join between $X$ and $Y$. Recall that the inner join consists of all the selected columns from the concatenated rows $X[i],Y[j]$ where  the joined on columns of the rows $X[i]$ and $Y[j]$ are equal. Alternatively, an inner join can be thought of as the intersection between the joined on columns of $X$ and $Y$. 

By construction, if row $X[i]$ has a matching row in $Y$ then this row will occupy at most one of the rows ${\widetilde{Y}^1}[i],{\widetilde{Y}^2}[i],{\widetilde{Y}^3}[i]$. To determine which the parties input the secret shared of these rows to an MPC protocol where the join on columns are compared. In particular, for each $i$ and rows ${\widetilde{Y}^1}[i],{\widetilde{Y}^2}[i],{\widetilde{Y}^3}[i]$ the bits $b_1[i],b_2[i],b_3[i]$ are generated where $b_l[i]=1$ iff the joined on columns of ${\widetilde{Y}^1}[i]$ are equal to that of $X[i]$. The MPC circuit then computes $b[i]=b_1[i]\oplus b_2[i]\oplus b_3[i]$ and $Y'[i]=b_1[i]{\widetilde{Y}^1}[i]\oplus b_2[i]{\widetilde{Y}^2}[i]\oplus b_3[i]{\widetilde{Y}^3}[i]$. $b[i]$ encodes whether the $i$th row of the output table is valid. If it is valid, then $Y'[i]$ is the row of $Y$ which matches $X[i]$.

The next phase of the join protocol is to compute the \texttt{where} clause of the query and any addition computation which is specified by the \texttt{select} clause. The \texttt{where} clause further filters the output table as a function of $Y'[i]$ and $X[i]$. For example, the query may specify that only rows where $Y_2'[i] + X_3[i] > 22$  are to be select. Regardless of the exact where clause, the MPC protocol will update the valid bit as $b[i] := b[i] \wedge f(Y'[i], X[i])$ where $f: \mathcal{Y}\times \mathcal{X} \rightarrow \{0,1\}$ is the filtering function specified by the \texttt{where} clause.

Finally, the additional computation specified by the \texttt{select} query is performed. Specifically, the columns of the output table can either be directly copied from the input tables $X,Y$ or can be a function of the given row. For example, the query could be of the form 
$$
\texttt{select } X_1, Y_2 + X_3 \text{ from } X \texttt{ inner join } Y \texttt{ on } X_1 = Y_1
$$
In this case the first column of the output will be $X_1$ while the second column will consist of the second column of $Y$ plus the third column of $X$. In general we can view the \texttt{select} clause as a function which takes the two rows $X[i]$ and $Y'[i]$ and computes a new row with the specified output columns. 

Several optimizations can be applied to this protocol. First, observe that only columns of $Y$ which explicitly appear in the query need to be input to the switching networks. This reduces the amount of data to be sent and improves performance. Secondly, when comparing the joined on columns, instead of computing the equality circuit between all of these columns it suffices to compare the randomized encodings. In the event that the joined on column(s) contain many bits, comparing the encodings can reduce the size of the equality circuit. In addition, observe that including columns from $X$ in the output table is essentially free due to these secret share columns simply being copied from $X$. Leveraging this the queries can be optimized by ensuring that the majority of the output columns are taken from $X$. Moreover, if a joined on column from $Y$ is in the \texttt{select} clause, this output column can be replaced with the matching column in $X$.

Also observe that the computation perform heavily lends itself to SIMD instructions. That is, the same computation is repeatedly applied to each row of the output table. Modern MPC protocol such as the ABY$^3$ framework \cite{aby3,highthroughput} are optimized for this setting and can process billions of binary circuit gates per second\cite{highthroughput}. In addition the ABY$^3$ framework can switch between using binary and arithmetic circuits based on which is most efficient for the given computation. 

\paragraph{Left/Right Join}

A left join query is similar to an inner join except that all of the rows from the left table $X$ are included in the output. In this case all of the output fields that are a function of the right table but do not have a matching row from the right table are initialized as a default value, typically \texttt{NULL}.  This type of join has the property that all of the output rows maintain their validity\footnote{More specifically, the output rows copy the validity bit $b[i]$ from the $X$ table.} and the valid bits $b[i]$ need not be updated. Instead the bits $b[i]$ as computed in the inner join are used to determine if the fields that are a function of $Y$ need to be initialized to the default value. That is, all rows that are in the inner join are computed as before and the remaining rows from $X$ use the bit $b[i]$ to initialize the missing values to the default. A right join can be implemented by swapping $X$ an $Y$.

\paragraph{Union and Set Minus}

Our framework is also capable of computing the union of two table. Specifically, we define the union operator as taking all of the rows from the left table and all of the rows from the right table which would not be present in the inner join. Note that the union operator assumes that the schemes of the two tables are identical. Somewhat surprisingly the protocol implementing union is conceptually similar to the inner join. The key observation is that the inner join computes a valid bit for each row of $X$ which encodes whether that row is in the intersection. By flipping that bit\footnote{Note that the special case where a row of $X$ is not valid as opposed to being outside the intersection must be handled.}, it now encodes whether that row is in $X\backslash Y$. Given such an operator the union of $X$ and $Y$ can trivially be constructed as $(X\backslash Y) || Y$ where the $||$ operator denotes the row-wise concatenation of $Y$ to the end of $X\backslash Y$.


\paragraph{Full Join}

A full join can be viewed as a hybrid between an inner join and a union. Specifically, the output table contains the inner of $X$ and $Y$ which is then unioned with $X$ and $Y$ on the right side. Put another way, a full join consists of the inner join where the missing rows from $X,Y$ additionally included. 

In the case that $X$ or $Y$ contain invalid rows, then we construct a full join protocol using two other join operations. First a left join between $X$ and $Y$ is performed. As a result the rows in the inner join are correctly compute and all records from $X$ are contained in the output table. What remains is to add all of the missing rows from $Y$ which is achieved by taking the union with the output table and $Y$. The overhead of this protocol is effectually twice that of the other protocol.

We note that under some restrictions on the tables being joined, a more efficient protocol for full joins can be achieved. We defer an explanation of this technique to \sectionref{sec:threatlog}.

%
%
%Next, we introduce a second protocol that requires $X$ and $Y$ to contain only valid rows.
%This join is implemented in two phases. First the union of just the joined on columns is performed, denoted as $Z=X\cup Y$. The parties will then generate randomized encodings for $Z,X,Y$. The encodings for $Z$ will be revealed to all parties while $X,Y$ are respectively revealed to party 0,1. Importantly, before these encoding are generated, the rows of the $Z$ table must be obliviously shuffled. This ensure that ordering information between $Z$ and $X,Y$ does not leak information. 
%
%Given these encodings, party 0 can use an oblivious permutation to reorder the rows of $X$ such that the encodings of $X$ and $Z$ are aligned. Similarly, party 1 can perform the same operation for $Y$. Let these new tables be denoted as $X',Y'$ respectively. In particular, the encodings for row $X'[i]$ should The final full join table $Z'$ can then be constructed by applying the select and where clause to each concatenated row $X'[i] || Y'[i]$.
%
%Several optimizations can be applied to this strategy. First, the $Z$ tables is computed the joined on columns of $X,Y$ can be compressed to a maximum of $\sigma$ bits by multiplying them with a random linear matrix as was performed in \sectionref{sec:join}. As a result, less data needs to be joined on in the union protocol resulting in improved performance. When the encodings for $X,Y$ are later computed, these compressed values can be reused. Secondly, in the case that there are several tables $U,...,X,Y$ all being full joined, the protocol can achieve even better performance. First, the union of the joined on columns of these tables is computed as before. Let this table be denoted as $Z$.  The original
%
%


%\subsection{Cardinality Revealing Joins}
%
%In some situations it may be possible to allow the cardinality of the join to be revealed to the parties. This is turn can be leveraged in conjunction with the uniqueness property of the joined on columns to improve the efficiency of the protocols.  Consider computing the inner join with an intersection size of $n'$. This immediately implies that the parties know $n-n'$ of the output table are invalid rows. However, which rows these are can convey significant information about the data that is contain. Say table $X$ is joined with $A,B,C,D,...$ and for all of these joins the first row of $X$ is always in the inner join. All of the parties now know that the first row of $X$ is contained in each of these other tables. For example, consider the $X$ being hospital patients, $A$ containing the last admission date and $B$ containing a list of patients with HIV/AIDS. When combined with other information, learning that the first row of $X$ was admitted on some day $A[j]$ and has HIV/AIDS could potentially reveal who this person is. 
%
%To prevent the parties correlating information in this way we introduce a new set of join protocols specifically designed for this setting. The core technique is for party 0 and 1 to sample random permutation networks for each table. After all permutations are applied, the row order of both tables are uniformly distributed in the view of the parties. The randomized encoding phase can then be performed where party 0 learns the encodings for $X$ and party 1 learns them for $Y$. Instead of proceeding with the construction of the cuckoo table, the parties can directly exchange the encodings. This immediately reveals which rows are in the intersection and allows the parties to construct the output table from the randomized $X,Y$ tables by directly copying the appropriate secret shares. This technique can be extended to the other types of joins listed above. In all cases the knowledge of the randomized encodings allow the parties to appropriately construct the output table. We note that this approach is similar to \cite{LTW13}
%

\subsection{Non-unique Join on Column}


When the uniqueness of the columns being joined on does not hold the security guarantees begin to erode. Recall that the randomized encodings for $X,Y$ are revealed to party 0,1 respectively and therefore some of these encodings will be duplicates due to the joined on columns not containing unique values. Learning the distribution of these duplicates reveals that the underlaying table has the same distribution. In the event that only one of the tables contains duplicates, the core protocol can naturally be extended to compute the various join operations subject to party 0 learning the duplicate distribution. This is achieved be requiring the left table $X$ contain the duplicates rows. After learning the randomized encodings for this table party 0 can program the switching networks appropriately to query the duplicate locations in the cuckoo hash table. 

When both tables contain duplicates we fall back to the less secure protocol architecture of Laur et al.\cite{LTW13}. In particular, this style of protocol  performs an oblivious shuffling of the table rows and then reveals all of the randomized encodings to all of the parties. Given this information the parties can construct the desired join. We suggest that the performance of these two primitive can be improved over \cite{LTW13} by 1) implementing the random shuffle using two random permutation networks from \sectionref{sec:switch} where party 0 and 1 both privately sample one of the permutations. 2) Replace the use of AES with our improved randomized encodings (LowMC and random binary matrix). Given these optimization the overhead of these protocols should be comparable to our standard join techniques. The major shortcoming of this approach is that the duplicate distribution and the size of join is revealed to all of the parties. As discussed in the related work section, this can limit several important application such as threat log comparison. 

\subsection{Revealing Results}

Revealing a table in general requires two operations. First observe that the data in the invalid rows is not cleared out by the join protocols. As such naively reconstructing these rows would lead to significant leakage. Instead the bitwise \textsc{and} between the valid bit and the row is computed. This ensures that all invalid rows have a deterministic value and therefore can be simulated. The second operation is to perform an oblivious shuffle of the rows. This operation randomly reorders all the rows without revealing the ordering to any of the parties. The necessity of this operation stems from the  row ordering (prior to the shuffle) being input dependent. Moreover, this ordering can be used to correlate between two different output tables. In particular, the row ordering of say $X\cap Y$ and $X\cap Z$ will be the same up to some rows being valid while other may not. In the case of sets this ordering does not reveal additional information since it can be inferred given the ideal output. However, this is not true in general when the items being joined are key-value pairs. 


\section{Computing a Function of a Table}

In addition to join queries, our framework can perform computation on a single table. For example, selecting $X_1+X_2$  where $X_3>42$. This type of select statement can easily be express with any generic MPC protocol, in particular \cite{aby3, highthroughput}. The key property is that all of the operations are with respect to a single row of $X$. As such, this computation can be expressed as a circuit and evaluated in parallel on each row. 

Our framework also considers a second class of functions on a table that allow computation between rows. For exampled, computing the sum of a column. We refer to this broad class of operations as an aggregation function. Depending on the exact computation, various levels of efficiencies can be achieved. Our primary approach is to employ the ABY$^3$ framework \cite{aby3} to express the desired computation in the most efficient way possible and then to evaluate the resulting circuit. Next we highlight a sampling of some important aggregation operations:
\begin{itemize}
	\item Sum: In this case there is a vector of secret shared data that need to be added together.  The default secret sharing format that we employ is a binary sharing. First all $n$ values are converted to an arithmetic sharing which requires $n\ell$ binary gates \cite{aby3}. The parties can then locally sum the arithmetic shares without any additional communication. This approach requires $\ell$ rounds of communication. One important note is that each row must be check to ensure that it is value. We propose doing this by computing the logical AND between the valid bit and the value being summed before it is converted to an arithmetic share. Given this, the correct result is the sum of the arithmetic shares, regardless of the validity of any given row.
	
	\item Count/Cardinality: Here, we consider two cases. 1) In the general case we also assume that the actual secret shared table containing the join is desired. Here, the count/cardinality can be computed in a similar way as sum except that the summation is directly over the valid bits. The more efficient bit injection protocol\cite{aby3} can be used to convert each bit to an arithmetic sharing in a constant number of rounds. In particular, the malicious secure bit injection protocol provided in \cite{aby3} is suggested due to it reducing the overall communication. 
	
	2) In the case that only the cardinality is desired, a more efficient protocol can be applied. First, assume that the LowMC encoding procedure is used. Then the parties can first generate the randomized encodings but not reveal the secret shares to any party. Each of these two sets of shared encodings are then obviously shuffled and revealed to the parties which should learn the cardinality. In the Diffie-Hellman encodings are used, a straight forward adaption of the protocol of De Cristofaro  et. al \cite{DBLP:conf/cans/CristofaroGT12} can be used.
	
	\item  Min/Max: Unlike computing sums, logical operations such as max are most efficiently computed over using a binary sharing format. In particular, we propose a binary tree structure where elements are compared and the min/max item is propagated. Concerning invalid rows, these values can be initialized to the maximum or minimum value to guarantee that the other value will be propagated. The overall complexity of this approach is $O(n\ell)$ binary gates and $O(\ell log n)$ rounds.
\end{itemize}

\section{Beyond Three Parties}

In some settings where many parties are providing tables to be computed on the three party requirement with at most one corruption may not be acceptable. In particular, a larger corruption threshold such as 3-out-of-5 or 4-out-of-7 may be desirable. Our protocol can naturally be extended to these setting with some caveats. First, the binary circuit based MPC protocol\cite{highthroughput} which our standard three party protocol relies on can naturally be extended to these larger (honest majority) corruption thresholds. Alternatively, several other suitable protocol such as\cite{...} have been proposed. This change immediately implies that the no minority set of corrupted parties can decrypt/reconstruct the secret shared tabled. However, our three party protocol also requires party 0 and 1 learning the randomized encodings in the clear.  As has previous been described, when the adversary learns both sets of randomized encodings they can infer information about the cardinality of the join. As such, our protocol offers two levels of security. When the adversary corrupts parties 0 and 1 the cardinality of the join is revealed. Otherwise, if an honest majority is present then the view of the corrupted parties can be simulated in the semi-honest setting. 




\input{protocol-figure}
\section{Omitted Proofs}

\subsection{Permutation Network}\label{sec:perm-proof}

We now formally prove that the oblivious permutation network protocol in \figureref{fig:switching-net}	and repeated in \figureref{fig:perm-net-repeat} is secure with respect to the  \f{permute} functionality of  \figureref{fig:perm-ideal-2}.

\begin{figure}
	\framebox{\begin{minipage}{0.95\linewidth}\small
			Parameters: $3$ parties denoted as \programmer, \sender and \receiver. Elements are strings in $\Sigma:=\{0,1\}^\sigma$. An input, output vector size of $n, m$.
			\smallskip			
			
			{\bf [Permute]} Upon the command $(\textsc{Permute}, \pi, \shareTwo{A}_0)$ from  \programmer and $(\textsc{Permute}, \shareTwo{A}_1)$ from  \sender. Require that $\pi: [m]\rightarrow [n]$ is \emph{injective} and  $\shareTwo{A}_0,\shareTwo{A}_1\in \Sigma^{n}$. Then:
	\begin{enumerate}[leftmargin=.5cm]
		\item  \programmer uniformly samples a bijection $\pi_0 : [n]\rightarrow[n]$ and  let $\pi_1 :[n] \rightarrow[m]$ s.t. $\pi_1\circ \pi_0 = \pi$.  \programmer sends $\pi_0 $ and  $S\gets \Sigma^{n}$  to  \sender.
		\item  \sender sends $B := ( \shareTwo{A_{\pi_0(1)}}_1 \oplus S_1, ...,  \shareTwo{A_{\pi_0(n)}}_1 \oplus S_n)$ to  \receiver.
		\item  \programmer sends $\pi_1$ and $T\gets\Sigma^{m}$ to  \receiver who outputs $\shareTwo{A'}_0:=\{B_{\pi_1(1)} \oplus T_1,...,B_{\pi_1(m)}\oplus T_m\}$.  \programmer outputs $\shareTwo{A'}_1:=\{ S_{\pi_1(1)}\oplus T_1\oplus  \shareTwo{A_{\pi(1)}}_0,...,S_{\pi_1(m)}\oplus T_m\oplus \shareTwo{A_{\pi(m)}}_0\}$.
	\end{enumerate}
	\end{minipage}}
	\caption{The Oblivious Permutation Network protocol $\proto{permute}$ repeated. }
	\label{fig:perm-net-repeat}	
\end{figure}

\begin{figure}\small
	\framebox{\begin{minipage}{0.95\linewidth}
			Parameters: $3$ parties denoted as the \programmer, \sender and \receiver. Elements are strings in $\Sigma:=\{0,1\}^\sigma$. An input vector size of $n$ and output size of $m$.
			
			{\bf [Permute]} Upon the command $(\textsc{Permute}, \pi, \shareTwo{A}_0)$ from the \programmer and $(\textsc{Permute}, \shareTwo{A}_1)$ from the \sender:
			\begin{enumerate}
				\item Interpret $\pi: [m]\rightarrow [n]$ as an injective function and $A\in \Sigma^n$. 
				\item Compute $A'\in \Sigma^m$ s.t. $\forall i\in [m], A_{\pi(i)} = A'_i$.
				\item Generate $\shareTwo{A'}$ and send $\shareTwo{A'}_0$ to \programmer and $\shareTwo{A'}_1$ to \receiver.
			\end{enumerate}
	\end{minipage}}
	\caption{The Oblivious Permutation Network ideal functionality \f{permute}.}
	\label{fig:perm-ideal-2}	
\end{figure}

\begin{theorem}\label{thm:permute}
	Protocol $\proto{permute}$ of \figureref{fig:perm-net-repeat} securely realized the ideal functionality \f{permute} of \figureref{fig:perm-ideal-2} given at most one party is corrupted in the semi-honest model.
\end{theorem}
\begin{proof}
	Correctness follows directly from $\pi_1\circ \pi_0 = \pi$ and that the masks cancel out.
	With respect to simulation, consider the following three cases:
	\begin{enumerate}
		\item \emph{Corrupt \programmer}: The view of \programmer contains no messages and therefore is trivial to simulation. 
		\item \emph{Corrupt \sender}:  The view of \programmer contains $\pi_1, S$ which are sent by \programmer. The simulator can uniformly sample $\pi_1:[m]\rightarrow[n]$ from all such injective functions and uniformly sample $S\gets\Sigma^n$. Clearly $S$ has the same distribution.
		
		With respect to $\pi_1$, observe if $\pi_1$ if first fixed uniformly at random then there are exactly $(n-m)!$ ways to choose $\pi_0$. Moreover, for each choice of $\pi_1$ there is a disjoint set of possible $\pi_0$ values. Therefore, \programmer sampling $\pi_0$ uniformly at random results in the distribution of $\pi_1$ also being uniform.
		
		%consider the following hybrid: \programmer first uniformly sample $\pi_1:[m]\rightarrow[n]$ and then defines $\pi_0:[n]\rightarrow[n]$ appropriately. For each choice of $\pi_1$, there are always exactly ${n\choose m}$ options for $\pi_0$. What is more, these options are unique to this choice of $\pi_1$.
		\item \emph{Corrupt \receiver}: The view of \receiver contains $B:= ( A_{\pi_0(1)} \oplus S_1, ..., A_{\pi_0(n)} \oplus S_n)$  and $\pi_1, T\in \Sigma^m$. $\pi_1,T$ are sampled uniformly and therefore trivial to simulation. similarly, each $B_i=A_{\pi_0(i)}\oplus S_i$ where $S_i$ is uniformly distributed in their view. Therefore $B_i$ is similarly distributed. 
	\end{enumerate}
\end{proof}


\subsection{Duplication Network}\label{sec:dup-proof}


We now formally prove that the oblivious duplication network protocol in \figureref{fig:switching-net}	and repeated in \figureref{fig:perm-net-repeat} is secure with respect to the  \f{dup} functionality of  \figureref{fig:dup-ideal-2}.

\begin{figure}
	\framebox{\begin{minipage}{0.95\linewidth}\small
			Parameters: $3$ parties denoted as \programmer, \sender and \receiver. Elements are strings in $\Sigma:=\{0,1\}^\sigma$. An input, output vector size of $n$.
			\smallskip
			
			

{\bf [Duplicate]} Upon the command $(\textsc{Duplicate}, \pi, \shareTwo{A}_0)$ from  \programmer and $(\textsc{Duplicate}, \shareTwo{A}_1)$ from  \sender. Require that $\pi: [n]\rightarrow [n]$ s.t $\pi(1)=1$ and $ \pi(i)\in \{i,\pi(i-1)\}$ for $i\in [2,n]$ and  $A\in\Sigma^{n}$. Then:
\begin{enumerate}[leftmargin=.5cm]
	
	\item  \programmer  computes the vector $b\in\{0,1\}^{m}$ such that $b_1=0$ and for $i\in[2,n],$ $b_i=1$ if $\pi(i)=\pi(i-1)$ and 0 otherwise.
	
	\item \sender samples $\shareTwo{B}_1, W^0,W^1\gets \Sigma^{n}, \shareTwo{B_1}_0\gets\Sigma$ and $\phi\gets\{0,1\}^n$. \sender redefine $\shareTwo{B_1}_1:=\shareTwo{A_1}_1\oplus \shareTwo{B_1}_0$. For $i\in [2,n]$, \sender sends 
	\begin{align*}
	M^0_i&:= \shareTwo{A_i}_1 \ \ \ \oplus \shareTwo{B_i}_1 \oplus W^{\phi_i}_i\\
	M^1_i&:= \shareTwo{B_{i-1}}_1 \oplus \shareTwo{B_i}_1 \oplus W^{\overline{\phi_i}}_i
	\end{align*}
	and $\shareTwo{B_1}_0,\phi$ to  \programmer. \sender sends $\shareTwo{B}_1,W^0, W^1$ to  \receiver.  
	\item\programmer sends $\rho:=\phi\oplus b, R\gets\Sigma^n$ to  \receiver who responds with $\{ W^{\rho_i}_i : i\in [2,n] \}$. For $i\in [2,n]$, \programmer defines
	$$
		\shareTwo{B_i}_0:= M^{b_i}_i \oplus W^{\rho_i}_i\oplus b_i\shareTwo{B_{i-1}}_0
	$$
	\programmer outputs $\shareTwo{A'}_0:=\shareTwo{B}_0\oplus R\oplus \pi(\shareTwo{A}_0)$ and \receiver outputs $\shareTwo{A'}_1:=\shareTwo{B}_1\oplus R$.
\end{enumerate}
	\end{minipage}}
	\caption{The Oblivious Duplication Network protocol \proto{duplicate} repeated. }
	\label{fig:dup-net-repeat}	
\end{figure}


\begin{figure}\small
	\framebox{\begin{minipage}{0.95\linewidth}
			Parameters: $3$ parties denoted as the \programmer, \sender and \receiver. Elements are strings in $\Sigma:=\{0,1\}^\sigma$. An input vector size of $n$ and output size of $n$.
			
			{\bf [Duplicate]} Upon the command $(\textsc{Duplicate}, \pi, \shareTwo{A}_0)$ from the \programmer and $(\textsc{Duplicate}, \shareTwo{A}_1)$ from the \sender:
			\begin{enumerate}
				\item Interpret $\pi: [n]\rightarrow [n]$ as a function s.t. $\pi(1)=1, \pi(i)\in \{i,\pi(i-1)\}$ for $i\in[2,n]$ and $A\in \Sigma^n$. 
				\item Compute $A'\in \Sigma^m$ s.t. $\forall i\in [n], A_{\pi(i)} = A'_i$.
				\item Generate $\shareTwo{A'}$ and send $\shareTwo{A'}_0$ to \programmer and $\shareTwo{A'}_1$ to \receiver.
			\end{enumerate}
	\end{minipage}}
	\caption{The Oblivious Duplication Network ideal functionality \f{duplicate}.}
	\label{fig:dup-ideal-2}	
\end{figure}


\begin{theorem}\label{thm:dup}
	Protocol $\proto{duplicate}$ of \figureref{fig:dup-net-repeat} securely realized the ideal functionality \f{duplicate} of \figureref{fig:dup-ideal-2} given at most one party is corrupted in the semi-honest model.
\end{theorem}
\begin{proof}
	Correctness follows an inductive argument. It is easy to verify $B_1=\shareTwo{A_1}_1$ and that this is correct since $\pi(1)=1$ by definition. Inductively let us assume that ${B_{i-1}}=\shareTwo{A_{\pi(i-1)}}_1$ and we will show that ${B_i}=\shareTwo{A_{\pi(i)}}_1$.  
	Observe that for $i\in[2,n]$
	\begin{align*}
		B_i&=\shareTwo{B_i}_0 \oplus \shareTwo{B_i}_1\\
		    &=(M^{b_i}_i\qquad\qquad\qquad\qquad\qquad\qquad\qquad\ \ \  \oplus W^{\rho_i}_i\oplus b_i\shareTwo{B_{i-1}}_0) \oplus (\shareTwo{B_i}_1)\\
 		    &=(\overline{b_i}\shareTwo{A_i}_0\oplus b_i\shareTwo{B_{i-1}}_1 \oplus \shareTwo{B_{i}}_1\oplus W^{b_i\oplus\phi_i}_i\oplus W^{\rho_i}_i\oplus b_i\shareTwo{B_{i-1}}_0) \oplus (\shareTwo{B_i}_1) \\
		    &=\overline{b_i}\shareTwo{A_i}_0 \oplus b_i B_{i-1}
	\end{align*}
	And therefore $B=\pi(\shareTwo{A}_1)$ and
	
	\begin{align*}
		A'=&\shareTwo{B}_1\oplus R \oplus \pi(\shareTwo{A}_0) \oplus \shareTwo{B}_1\oplus R\\
		=& B \oplus \pi(\shareTwo{A}_0)\\
		=& \pi(\shareTwo{A}_1) \oplus \pi(\shareTwo{A}_0)\\
		=& \pi(A)\\
	\end{align*}
	
	With respect to simulation, consider the following three cases:
	\begin{enumerate}
		\item \emph{Corrupt \programmer}: The transcript of \programmer contains $M^0, M^1\in \Sigma^n,\shareTwo{B_1}_0\in \Sigma, \phi\in \{0,1\}^n$ from \sender and $W^{b_i\oplus\phi_i}_i$ from \receiver. First observe that $\shareTwo{B_1}_0, \phi$ are sampled uniformly and therefore can be simulated as the same. 
		
		 Next recall that
	\begin{align*}	
	M^{b_i}_i=&...\oplus\shareTwo{B_i}_1\\
	M^{\overline{b_i}}_i=&...\oplus W^{\overline{b_i\oplus\phi_i}}_i	
	\end{align*}
	where $\shareTwo{B_i}_1,  W^{\overline{b_i\oplus\phi_i}}_i\in \Sigma$ are sampled uniformly can not in the view of \programmer.  Therefore $M^0_i,M^1_i$ are distributed uniformly.
	
		\item \emph{Corrupt \sender}:  The transcript of \sender contains nothing and therefore is trivial to simulate. Note that the distribution of the output shares in independent of \sender's random tape (view) due to $\programmer,\receiver$ re-randomizing the shares with $R\gets\Sigma^n$.
		
		\item \emph{Corrupt \receiver}:  The transcript of \receiver contains $\shareTwo{B_1}_1, W^0,W^1$ from \sender and $\rho$ from \programmer. $W^0,W^1$ are sampled uniformly and therefore can be simulated as the same. $\shareTwo{B_1}_1=A_1\oplus \shareTwo{B_1}_0$ where $\shareTwo{B_1}_0$ is sampled uniformly and not in the view. Therefore $\shareTwo{B_1}_1$ is distributed uniformly. The same applies to $\rho$ since $\phi$ is uniform and not in the view. 
	\end{enumerate}
\end{proof}




\subsection{Switching Network}\label{sec:switch-proof}


We now formally prove that the oblivious switching network protocol in \figureref{fig:switching-net} and repeated in \figureref{fig:switch-net-repeat} is secure with respect to the  \f{switch} functionality of  \figureref{fig:switch-ideal-2}. In the proof we will replace calls to the Permutaiton and Duplication protocols of \proto{Switch} with their ideal functionalities (\figureref{fig:perm-ideal-2}, \ref{fig:dup-ideal-2}).


\begin{figure}
	\framebox{\begin{minipage}{0.95\linewidth}\small
			Parameters: $3$ parties denoted as \programmer, \sender and \receiver. Elements are strings in $\Sigma:=\{0,1\}^\sigma$. An input, output vector size of $n, m$.
			\smallskip
			
			
				{\bf [Switch]} Upon the command $(\textsc{Switch}, \pi, \shareTwo{A}_0)$ from  \programmer and $(\textsc{Switch}, \shareTwo{A}_1)$ from  \sender where $\pi: [m]\rightarrow [n]$ and $\shareTwo{A}_0,\shareTwo{A}_1\in \Sigma^{n}$. 
				\begin{enumerate}[leftmargin=.5cm]
					
					\item  \programmer samples an injection $\pi_1:[m]\rightarrow [n]$ s.t. for $i\in image(\pi)$ and $k=|preimage(\pi, i)|$,  $\exists j$ where $\pi_1(j)=i$ and $\{\pi_1(j+1), ...,\pi_1(j+k) \}\cap image(\pi)=\emptyset$. 
					 \programmer  sends $(\textsc{Permute}, \pi_1, \shareTwo{A}_0)$ to \f{Permute} and  \sender sends $(\textsc{Permute}, \shareTwo{A}_1)$.  \programmer receives $\shareTwo{B}_{0}\in \Sigma^{m}$ in response and  \receiver receives $\shareTwo{B}_{1}\in \Sigma^{m}$. 
					
					\item  \programmer defines $\pi_2:[m]\rightarrow[m]$ s.t. for $i\in image(\pi)$ and $k:=|preimage(\pi, i)|$ and $j$ where $\pi_1(j)=i$, then $\pi_2(j)=...=\pi_2(j+k)=j$. \programmer and \receiver respectively send $(\textsc{Duplicate}, \pi_2, \shareTwo{B}_0)$ and $(\textsc{Duplicate}, \shareTwo{B}_{1})$ to \f{Duplicate}. As a result \programmer obtains $\shareTwo{C}_0\in \Sigma^m$ from \f{Duplicate} and \sender obtains $\shareTwo{C}_1\in \Sigma^m$.
					
					\item \programmer computes the permutation $\pi_3:[m]\rightarrow[m]$ such that for  $i\in image(\pi)$ and $k=|preimage(\pi, i)|$, $\{\pi_3(\ell) : \ell\in preimage(\pi, i)\}=\{j, ..., j +k\}$ where $i=\pi_1(j)$.	 \programmer sends $(\textsc{Permute}, \pi_3, \shareTwo{C}_0)$ to \f{Permute} and  \sender sends $(\textsc{Permute},\shareTwo{C}_1)$.  \programmer receives $S\in \Sigma^{m}$ in response. \programmer and \receiver respectively receives and outputs $\shareTwo{A'}_0,\shareTwo{A'}_1\in \Sigma^{m }$.
				\end{enumerate}
	\end{minipage}}
	\caption{The Oblivious Switching Network protocol \proto{switch} repeated. }
	\label{fig:switch-net-repeat}	
\end{figure}

\begin{figure}\small
	\framebox{\begin{minipage}{0.95\linewidth}			
			
			Parameters: $3$ parties denoted as the \programmer, \sender and \receiver. Elements are strings in $\Sigma:=\{0,1\}^\sigma$. An input vector size of $n$ and output size of $m$.
			
			{\bf [Switch]} Upon the command $(\textsc{switch}, \pi, \shareTwo{A}_0)$ from the \programmer and $(\textsc{switch}, \shareTwo{A}_1)$ from the \sender:
			\begin{enumerate}
				\item Interpret $\pi: [m]\rightarrow [n]$ and $A\in \Sigma^n$. 
				\item Compute $A'\in \Sigma^m$ s.t. $\forall i\in [m], A_{\pi(i)} = A'_i$.
				\item Generate $\shareTwo{A'}$ and send $\shareTwo{A'}_0$ to \programmer and $\shareTwo{A'}_1$ to \receiver.
			\end{enumerate}

	\end{minipage}}
	\caption{The Oblivious Switching Network ideal functionality \f{switch} repeated.}
	\label{fig:switch-ideal-2}	
\end{figure}



\begin{theorem}\label{thm:switch}
	Protocol $\proto{Switch}$ of \figureref{fig:switch-net-repeat} securely realized the ideal functionality \f{switch} of \figureref{fig:switch-ideal-2} given at most one party is corrupted in the semi-honest  model.
\end{theorem}
\begin{proof}
	Correctness follows from the first oblivious permutation call rearranges the input vector such that each output item which appears $k$ times is followed by $k-1$ items which do not appear in the output. The duplication network then copies each of these output items into the next $k-1$ position. The final permutation places these items in the final order. 
	
	With respect to simulation, the transcript of each party contains their transcripts of three subprotocols: Permute, Shared-Duplicate and Shared-Permute. By \theoremref{thm:permute} the Permute subprotocol transcript can be simulated. Similarly, \theoremref{thm:shared} combined with \theoremref{thm:permute},\ref{thm:dup} also imply that the other two transcripts can be simulated. Therefore this implies that the overall protocol can be simulated given that no other communication is performed. 
		
\end{proof}
%
%
%\subsection{Shared Network}\label{sec:shared-proof}
%
%
%
%\begin{figure}
%	\framebox{\begin{minipage}{0.95\linewidth}\small
%			Parameters: $3$ parties denoted as \programmer, \sender and \receiver. Elements are strings in $\Sigma:=\{0,1\}^\sigma$. An input, output vector size of $n, m$.
%			\smallskip
%			
%			
%				{\bf [Shared]} Upon the command $(\textsc{Shared},cmd, \pi, \shareTwo{A}_0)$ from \programmer and $(\textsc{Shared}, cmd, \shareTwo{A}_1)$ from \sender. Require that $cmd\in \{\textsc{Permute},$ $\textsc{Duplicate},$ $\textsc{Switch}\}$, $\pi: [m]\rightarrow [n]$ and  $A\in \Sigma^{n}$.
%				\begin{enumerate}[leftmargin=.5cm]
%					\item \programmer sends $(cmd, \pi)$ to $\mathcal{F}_{cmd}$ and \sender sends $(cmd, \shareTwo{A}_1)$. 
%					\item In response, \programmer and \receiver respectively receive $\shareTwo{B}_0$ and $\shareTwo{B}_1$. \programmer outputs $\shareTwo{A'}_0:=\shareTwo{B}_0\oplus \pi(\shareTwo{A}_0)$ and \receiver outputs $\shareTwo{A'}_1:=\shareTwo{B}_1$.
%				\end{enumerate}
%	\end{minipage}}
%	\caption{The Oblivious Shared Switching Network protocol \proto{shared} repeated. }
%	\label{fig:shared-net-repeat}	
%\end{figure}
%
%\begin{figure}\small
%	\framebox{\begin{minipage}{0.95\linewidth}
%			
%			Parameters: $3$ parties denoted as the \programmer, \sender and \receiver. Elements are strings in $\Sigma:=\{0,1\}^\sigma$. An input vector size of $n$ and output size of $m$.
%			
%			{\bf [Shared]} Upon the command $(\textsc{Shared}, cmd, \pi, \shareTwo{A}_0)$ from the \programmer and $(\textsc{Shared},cmd, \shareTwo{A}_1)$ from the \sender:
%			\begin{enumerate}
%				\item Interpret $\pi: [m]\rightarrow [n]$ and $A=\shareTwo{A}_0\oplus \shareTwo{A}_1\in \Sigma^n$. 
%				\item If $cmd=\textsc{Permute}$, require $\pi$ to be injective.				
%				\item If $cmd=\textsc{Duplicate}$, require $n=m,\pi(1)=1, \pi(i)\in \{i, \pi(i-1)\}, \forall i\in[2,n]$. 
%				\item Require $cmd\in \{\textsc{Permute},\textsc{Duplicate},\textsc{Switch}\}$.
%
%				\item Compute $A'\in \Sigma^m$ s.t. $\forall i\in [m], A_{\pi(i)} = A'_i$.
%				\item Generate $\shareTwo{A'}$ and send $\shareTwo{A'}_0$ to \programmer and $\shareTwo{A'}_1$ to \receiver.
%				
%			\end{enumerate}
%			
%	\end{minipage}}
%	\caption{The Oblivious Shared Switching Network ideal functionality \f{shared}.}
%	\label{fig:shared-ideal-2}	
%\end{figure}
%
%
%
%\begin{theorem}\label{thm:shared}
%	Protocol $\proto{shared}$ of \figureref{fig:shared-net-repeat} securely realized the ideal functionality \f{shared} of \figureref{fig:shared-ideal-2} given at most one party is corrupted in the semi-honest  model.
%\end{theorem}
%\begin{proof}
%	Correctness follows from the correctness of the underlying subprotocol and that the \programmer can locally apply the mapping function to their own share. Simulation of this protocol is exactly that of the underlying subprotocol along with updating the output as specified. 
%\end{proof}
%




\subsection{Join Protocol}\label{sec:join-proof}



\begin{theorem}\label{thm:join}
	Protocol $\proto{join}$ of \figureref{fig:full_proto} securely realized the ideal functionality \f{join} of \figureref{fig:full_ideal} given at most one party is corrupted in the semi-honest \f{shared},\f{encode}-hybrid model with statistical security parameters $\lambda$.
\end{theorem}
\begin{proof}
\end{proof}




\section{Applications}\label{sec:app}
\subsection{Voter Registration}\label{sec:voter}

Improving the privacy and integrity of the United States voter registration system was a primary motivation of the developed protocols. In  the United States Electoral College, each state has the responsibility of maintaining their own list of registered citizens. A shortcoming of this distributed process is that without coordination between states it is possible for a voter to register in more than one state. If this person then went on to cast more than one vote the integrity of the system would be compromised. In the case of double registering, it is often a result of a person moving to a new state and failing to unregister from the old state. Alternatively, when a voter moves to a new state it may take them some time to register in the new state, and as such their vote may go uncast. The Pew Charitable Trust\cite{pew} reported 1 in 8 voter registration records in the United States contains a serious error while 1 in 4 eligible citizens remain unregistered. The goal in this application of our framework is to improve the accuracy of the voting registration data and help register eligible voters. 

A naive solution to this problem is to construct a centralized database of all the registered voters and citizen records. It is then a relatively straightforward process to identify persons with inaccurate records, attempt to double register or are simply not register at all. However, the construction of such a centralized repository of information has long had strong opposition in the United States due to concerns of data privacy and  excessive government overreach. As a compromise many states have volunteered to join the Electronic Registration Information Center (ERIC)\cite{eric} which is a non-profit organization with the mission of assisting states to improve the accuracy of America's voter rolls and increase access to voter registration for all eligible citizens. This organization acts as a semi-trusted third party which maintains a centralized database containing hashes of the relevant information, e.g. names, addresses, drivers license number and social security number. 

\iffullversion
\iffullversion
\else


\section{Summary of CCS 2020 Revision}

We have made significant efforts to address \emph{all} concerns that have been raised by the reviewers. The most significant changes include:
\begin{enumerate}
	\item Improvements in writing quality. We have rewritten/expanded certain sections which we believe lead to some confusion. The two most notable of these is the introduction and the oblivious switching network sections. The introduction now highlights the "outsourced secure computation setting" where the three parties act as servers that are oblivious to the data that they are computing on. We also give real-world motivations/examples for the setting. 
	
	\item A common comment was that the Duplication network section is confusing. As such, we have largely rewritten this section. The duplication network now has its own protocol box ($\proto{dup}$ of \figureref{fig:switching-net}) as opposed to being inlined into the switch net protocol. We also now start the explanation with a concrete example of the duplication protocol which is indented to give better intuition. This is followed by a new explanation of the "warmup" along with how to generalize it. A security proof of the protocol has been added to the appendix. 
	%	x  warmup was not clear. 
	%	x  Clearly state how to generalize (required!)
	%	x  missing proof of correctness/security (required!)
	
	\item The switching network section now starts with a concrete example that should aid understanding. We have also reworked the explanation in the text to be easier to read. Finally, we also give a comparison with two party computation switching network and permutation network based on Oblivious Transfer and Homomorphic Encryption.
	
	\item Some reviewers were confused by the fact that the oblivious permutation is actually an oblivious injection. We made this distinction clearer.
	
	\item We have added a discussion on microbenchmarks to the experiment section 6. This describes the relative amount of time each step takes. 
	
	
	\item We now more clearly compare with Google's Join-and-Compute protocol. This protocol was actually already compared with but we did not make it clear that it is the protocol that Google uses. We note that this protocol is significantly slower than ours and is not composable. We are not aware of any two party protocol that has an implementation and is composable.
	
	\item We added formal proofs of security for the core protocols (permutation, duplication, switch net, join protocol). These are in the appendix. 
	
	\item 
	One of the reviewers suggested we compared with VaultDB. We appreciate the suggestion and have cited the work but decided not to compare due to their protocol targeting a significantly different setting. Mapping our protocol into their setting would be an interesting future direction. 
	
	\item The ideal functionality and protocol boxes for the Join protocol have been reworked to improve readability. 
\end{enumerate}

Some other minor changes include:
\begin{enumerate}
\item  We clearly state of cuckoo hash table $m\approx1.5n$.
\item  We note that we are open source. 
\item  Fixed various typos.
\item  Include page numbers.
\item  Clearly state that KKRT can not be made composable without very large overhead. 
\end{enumerate}

\section{Voter Query Details}\label{sec:voterDetails}


\begin{figure*}[t!]\centering\footnotesize
	% \begin{figure*}[h]\centering
	\begin{tabular}{|l|| r | r |r |r|r|r||r|r | r |r |r|r|}
		\cline{2-13}
		\multicolumn{1}{c}{}         & \multicolumn{6}{|c||}{LAN}                                   & \multicolumn{6}{|c|}{WAN}                                    \\ \hline
		\multirow{2}{*}{Application} &                  \multicolumn{6}{c||}{$n$}                   &                   \multicolumn{6}{c|}{$n$}                   \\
		& $2^8$ & $2^{12}$ & $2^{16}$ & $2^{20}$ & $2^{24}$ & $2^{26}$ & $2^8$ & $2^{12}$ & $2^{16}$ & $2^{20}$ & $2^{24}$ & $2^{26}$ \\ \hline\hline
		Voter Intra-state            & 0.01  & 0.02     & 0.2      & 4.7      &    114.7 &  2,190.1 &   1.0 &      1.0 & 2.2      & 27.1     & 456.1    &  7,463.9 \\ \hline
		Voter Inter-state            & 0.01  & 0.02     & 0.3      & 7.0      &    134.8 &  2,546.4 &   1.6 &      1.6 & 4.0      & 45.4     & 747.7    & 12,284.1 \\ \hline
		Threat Log $N=2$          & 0.02  & 0.03     & 0.2      & 5.1      &    121.4 &    488.1 &   2.4 &      2.5 & 4.8      & 34.6     & 585.6    &  2,342.4 \\ \hline
		Threat Log $N=4$           & 0.05  & 0.09     & 0.9      & 17.9     &    388.4 &  1,553.9 &   6.6 &      6.8 & 13.1     & 108.7    & 1,739.2  &  6,956.8 \\ \hline
		Threat Log $N=8$          & 0.10  & 0.19     & 1.7      & 47.1     &  1,021.0 & 16,336.1 &  14.9 &     15.3 & 30.0     & 264.3    & 4,228.8  & 16,915.2 \\ \hline
	\end{tabular}
\vspace{-0.3cm}
	\caption{\label{fig:app} The running time in seconds for the Voter Registration and Threat Log applications. The input tables each contain $n$ rows.  }
	\vspace{-0.3cm}
\end{figure*}

Given the problem statement from \ref{sec:voter}, a naive solution is to construct a centralized database of all the registered voters and citizen records. It is then a relatively straightforward process to identify persons with inaccurate records, attempt to double register or are simply not register at all. However, the construction of such a centralized repository of information has long had strong opposition in the United States due to concerns of data privacy and  excessive government overreach. As a compromise many states have volunteered to join the Electronic Registration Information Center (ERIC)\cite{eric} which is a non-profit organization with the mission of assisting states to improve the accuracy of America’s voter rolls and increase access to voter registration for all eligible citizens. This organization acts as a semi-trusted third party which maintains a centralized database containing hashes of the relevant information, e.g. names, addresses, drivers license number and social security number. 
\fi

In particular, instead of storing this sensitive information in plaintext, all records are randomized using two cryptographically strong salted hash functions. Roughly speaking, before this sensitive information is sent to ERIC, each state is provided with the first salt value $salt_1$ and updates each value $v$ as $v := H(salt_1 || v)$. This hashed data is then sent to ERIC where the data is hashed a second time by ERIC which possesses the other salt value. The desired comparisons can then be applied to the hashed data inside ERIC's secure data center. When compared with existing alternative, this approach provides a moderate degree of protection. In particular, so long as the salt values remain inaccessible by the adversary, deanatomized any given record is likely non-trivial. However, a long series of works, e.g. \cite{deanon0,deanon1,deanon2,deanon3,deanon4}, have shown that a significant amount of information can be extracted with sophisticated statistical techniques. Moreover, should the adversary possess the salt values a straightforward dictionary attack can be applied.

We propose adding another layer of security with the deployment of our secure database join framework. In particular, two or more of the states and ERIC will participate in the MPC protocol. From here we consider two possible solutions. The first option is to maintain the existing repository but now have it secret shared between the computational parties. Alternatively, each state could be the long-term holder of their own data and the states perform all pairwise comparison amongst themselves. For reason of preferring the more distributed setting we further explore the pairwise comparison approach. 

The situation is further complicated by how this data is distributed within and between states. In the typical setting no single state organization has sufficient information to identify individuals which are incorrectly or double registered. For example, typical voter registration forms requires a name, home address and state ID/driver's license number. If two states compared this information there would be no reliable attribute for joining the two records. The name of the voter could be used but names are far from a unique identifier. The solution taken by ERIC is to first perform a join between a state's registered voters and their Department of Motor Vehicles (DMV) records, using the  state ID/driver's license number as the join-key. Since the DMV typically possesses an individual's Social Security Number (SSN), this can now be used as a unique identifier across all states. However, due to regulations within some states this join is only allowed to be performed on the hashed data or, presumably, on secret shared data.

In addition to identifying individuals that are double registered, the mission of ERIC is to generally improve the accuracy of all voter records. This includes identifying individuals that have moved and not yet registered in their new state or that have simply moved within a state and not updated their current address. In this case the joins between/within states should also include an indicator denoting that an individual has updated their address at a DMV which is different than the voter registration record. There are likely other scenarios which ERIC also identifies but we leave the exploration of them to future work.

Given the building blocks of \sectionref{sec:construction} it is a relatively straightforward task to perform the required joins. First a state performs a left join between their DMV data and the voter registration data. Within this join the addresses in the inner join are compared. In the event of a discrepancy, the date of when these addresses were obtained can be compared to identify the most up to date address. Moreover, the agency with the older address can be notified and initial a procedure to determine which, if any, of the addresses should be updated. 


Once this join is performed, each state holds a secret shared table of all their citizens that possess a state ID and their current registration status. Each pair of states can then run an inner join protocol using the social security number as the key. There are several cases that a result record can be in. First it is possible for a person to have a DMV record in two states and be registered in neither. The identity of these persons should not be revealed as this does not effect the voting process. The next case is that a person is registered in both states. We wish to reveal this group to both states so that the appropriate action can be taken. The final case that we are interested in is when a person is registered in state \emph{A} and has a newer DMV address in state \emph{B}. In this case we want to reveal the identity of the person to the state that they are registered to. This state can then contact the person to inquire whether they wish to switch their registration to the new state. 


This approach has additional advantages over the hashing technique of ERIC. First, all of the highly sensitive information such as a persons address, state ID number and SSN can still be hashed before being added to the database\footnote{The hashing originally performed by ERIC can be replaced with the randomized encoding protocol.}. However, now that the data is secret shared less sensitive information such as dates need not be hashed. This allows for the more expressive query described above which uses a numerical comparison. To achieve the the same functionality using the current ERIC approach these dates would have to be stored in plaintext which leaks significant information. In addition, when the ERIC approach performs these comparison the truth value for each party of the predicate is revealed. Our approach reveals no information about any intermediate value. 

\begin{figure}[ht]
	
	{
		\scriptsize
		\begin{align*}	
		stateA = \texttt{select }& DMV.name,\\
		& DMV.ID, \\
		& DMV.SSN, \\
		& DVM.date > Voter.date \ ?\\
		&\quad DMV.date : Voter.date \texttt{ as } date,\\
		& DVM.date > Voter.date \ ?\\
		&\quad  DMV.address : Voter.address \texttt{ as } address,\\
		& DVM.address \neq Voter.address \texttt{ as } mixedAddress, \\ 
		& Voter.name \neq \texttt{ NULL as } registered \\
		\texttt{ from } & DMV \texttt{ left join } Voter \\
		\texttt{ on } & DMV.ID = Voter.ID \\
		stateB = \texttt{select }&...\\
		resultA = \texttt{select } & stateA.SSN \\
		& stateA.address \texttt{ as } addressA\\
		& stateB.address \texttt{ as } addressB\\
		& stateA.registered \\
		& stateB.registered \\
		\texttt{ from } & stateA \texttt{ inner join } stateB \\
		\texttt{ on } & stateA.SSN = stateB.SSN\\
		\texttt{ where } & (stateA.date < stateB.date \texttt{ and } stateA.registered ) \\
		\texttt{ or } & (stateA.registered \qquad \qquad \ \, \texttt{ and }stateB.registered )\\
		resultB = \texttt{select } & ...
		\end{align*}
	}
	\caption{SQL styled join query for the ERIC voter registration application. \label{fig:voterQuery}}
\end{figure}

Once the parties construct the tables in \figureref{fig:voterQuery}, state \emph{A} can query the table $stateA$ to reveal all IDs and addresses where the $mixedAddress$ attribute is set to \texttt{true}. This reveals exactly the people who have conflicting addresses between that state's voter and DMV databases. When comparing voter registration data between states, state B should define $stateB$ in a symmetric manner as $stateA$. The table $resultA$ contains all of the records which are revealed to state \emph{A} and $resultB$, which is symmetrically defined, contains the results for state \emph{B}. We note that $resultA$ and $resultB$ can be constructed with only one join.


Both types of these queries can easily be performed in our secure framework. All of the conditional logic for the select and where clauses are implemented using a binary circuit immediately after the primary join protocol is performed. This has the effect that overhead of these operation is simply the size of the circuit which implements the logic times the number of potential rows contained in the output. 
%In particular, instead of storing this sensitive information in plaintext, all records are randomized using two cryptographically strong salted hash functions. Roughly speaking, before this sensitive information is sent to ERIC, each state is provided with the first salt value $salt_1$ and updates each value $v$ as $v := H(salt_1 || v)$. This hashed data is then sent to ERIC where the data is hashed a second time by ERIC which possesses the other salt value. The desired comparisons can then be applied to the hashed data inside ERIC's secure data center. When compared with existing alternative, this approach provides a moderate degree of protection. In particular, so long as the salt values remain inaccessible by the adversary, deanatomized any given record is likely non-trivial. However, a long series of works, e.g. \cite{deanon0,deanon1,deanon2,deanon3,deanon4}, have shown that a significant amount of information can be extracted with sophisticated statistical techniques. Moreover, should the adversary possess the salt values a straightforward dictionary attack can be applied.
%
%We propose adding another layer of security with the deployment of our secure database join framework. In particular, two or more of the states and ERIC will participate in the MPC protocol. From here we consider two possible solutions. The first option is to maintain the existing repository but now have it secret shared between the computational parties. Alternatively, each state could be the long-term holder of their own data and the states perform all pairwise comparison amongst themselves. For reason of preferring the more distributed setting we further explore the pairwise comparison approach. 
%
%The situation is further complicated by how this data is distributed within and between states. In the typical setting no single state organization has sufficient information to identify individuals which are incorrectly or double registered. For example, typical voter registration forms requires a name, home address and state ID/driver's license number. If two states compared this information there would be no reliable attribute for joining the two records. The name of the voter could be used but names are far from a unique identifier. The solution taken by ERIC is to first perform a join between a state's registered voters and their Department of Motor Vehicles (DMV) records, using the  state ID/driver's license number as the join-key. Since the DMV typically possesses an individual's Social Security Number (SSN), this can now be used as a unique identifier across all states. However, due to regulations within some states this join is only allowed to be performed on the hashed data or, presumably, on secret shared data.
%
%In addition to identifying individuals that are double registered, the mission of ERIC is to generally improve the accuracy of all voter records. This includes identifying individuals that have moved and not yet registered in their new state or that have simply moved within a state and not updated their current address. In this case the joins between/within states should also include an indicator denoting that an individual has updated their address at a DMV which is different than the voter registration record. There are likely other scenarios which ERIC also identifies but we leave the exploration of them to future work.
%
%Given the building blocks of \sectionref{sec:construction} it is a relatively straightforward task to perform the required joins. First a state performs a left join between their DMV data and the voter registration data. Within this join the addresses in the inner join are compared. In the event of a discrepancy, the date of when these addresses were obtained can be compared to identify the most up to date address. Moreover, the agency with the older address can be notified and initial a procedure to determine which, if any, of the addresses should be updated. 
%
%
%Once this join is performed, each state holds a secret shared table of all their citizens that possess a state ID and their current registration status. Each pair of states can then run an inner join protocol using the social security number as the key. There are several cases that a result record can be in. First it is possible for a person to have a DMV record in two states and be registered in neither. The identity of these persons should not be revealed as this does not effect the voting process. The next case is that a person is registered in both states. We wish to reveal this group to both states so that the appropriate action can be taken. The final case that we are interested in is when a person is registered in state \emph{A} and has a newer DMV address in state \emph{B}. In this case we want to reveal the identity of the person to the state that they are registered to. This state can then contact the person to inquire whether they wish to switch their registration to the new state. 
%
%
%
%
%This approach has additional advantages over the hashing technique of ERIC. First, all of the highly sensitive information such as a persons address, state ID number and SSN can still be hashed before being added to the database\footnote{The hashing originally performed by ERIC can be replaced with the randomized encoding protocol.}. However, now that the data is secret shared less sensitive information such as dates need not be hashed. This allows for the more expressive query described above which uses a numerical comparison. To achieve the the same functionality using the current ERIC approach these dates would have to be stored in plaintext which leaks significant information. In addition, when the ERIC approach performs these comparison the truth value for each party of the predicate is revealed. Our approach reveals no information about any intermediate value. 

\else

We propose adding another layer of security with the deployment of our secure database join framework. Within a single state, different agencies will first secret share their data to construct a join table containing the registration status of everyone within that state. This joined table can then be joined with the respective table from all of the other stated. In total, there would be 50 intra-state joins and then $50\times49$ inter-state joins. 

We envision that the intra-state join will be perform with ERIC and the state agencies as the participating parties. The inter-state joins can then be performed by ERIC and one of the agencies from each state. This ensures that the data remains secret shared at all times. The data that each state requires can then be revealed at the end of the computation. For a more detailed explanation of why these two types joins are required and the exact computation performed, see \appendixref{sec:voterDetails}.

\fi


The average US state has an approximate population of 5 million with about 4 million of that being of voting age. For this set size, our protocol is capable of performing the specified query in 30 seconds and 6GB of total communication. If we consider running the same query where one of the states is California with a voting population of 30 million, our protocol can identify the relevant records in five minutes. For a more extensive performance evaluation, see \sectionref{sec:eval}.


\subsection{Threat Log Comparison}\label{sec:threatlog}

Another motivating application is referred to as threat log comparison where multiple organizations share data about current attacks on their computer networks. The goal of sharing this data is to allow the participating parties to identify and stop threats in a more timely manner. Facebook has a service called ThreatExhange\cite{threat} which provides this functionality. One drawback of the Facebook approach is that all of the data is collected on their servers and is often viewable by the other participants. This architecture inherently relies on trusting Facebook with this data. 

We propose using our distributed protocol to provide a similar functionality while reducing the amount of trust in any single party, e.g. Facebook. In this setting we consider a moderate number of parties each holding a dataset containing the suspicious events on their network along with possible meta data on that event, e.g. how many times that event occurred. All of the parties input these sets into our join framework where the occurrences  of each event type are counted. An example of such an event is the IP address that makes a suspicious request. 

There are at least two ways to securely compute the occurrences of these events. One method is to perform a full join of all the events where the counts are added together during each join. The resulting table would contain all of the events and the number of times that each event occurred. The drawback of this approach is that each full joins require performing a left join followed by a union, twice the overhead compared to other join operations.

Now consider a different strategy for this problem.  First, the party can compute and reveal the union of the events. Given this information the parties can locally compute the number of times this event occurred on their network and secret share this information between the parties. The parties then add together this vector of secret shared counts and reveal it.

One shortcoming of this approach is no ability to limit which events are revealed. For example, it can be desirable to only reveal an event if it happens on $k$ out of the $n$ networks. This can achieved by having the parties compute and reveal the randomized encodings for all of the items in the union, instead of the items themselves. Under the same encoding key, each party holding a set employs the three server parties to compute the randomized encodings for the items in their set. These encodings are revealed to the party holding the set. For each encoding in the union, the parties use the MPC protocol to compute the number of occurrences that event had and conditionally reveal the value. For example, if at least $k$ of the networks observed the event. Other meta data and computation can also be performed as that stage.
















\section{Experiments}\label{sec:eval}


We implemented our full set of protocols and applications along with a performance evaluation of them here. They will be open source. We considered set intersection (without associated values), our various join and union operations, intersection cardinality, and intersection sum of key-value pairs along with the two application described in \sectionref{sec:app}. We also compare to protocols that offer a similar functionality, i.e. \cite{CCS:KKRT16, PSWW18,ASIACCS:BlaAgu12,DBLP:conf/cans/CristofaroGT12,cryptoeprint:2017:738}. Our implementation is written in c++ and building on primitives provided by \cite{libOTe}. Crucial to the performance of implementation is the widespread use of SIMD instructions that allow processing 128 binary gates with a throughput of one cycle.



\paragraph{Experimental Setup} We performed all of our experiments on a single server acquired in 2015 which is equipped with two 18 core CPUs at 2.7 GHz and 256 GB of RAM. Despite having many cores, our implementation restricts each party to a \emph{single thread}. We note this is a limitation of development time/resources and not of the protocols themselves. The parties communicate over a loopback device on the local area network which allows to shape the traffic flow to emulate a LAN and WAN setting. Specifically, the LAN setting allows 10 Gbps throughput with a latency of a quarter millisecond while the WAN setting allows an average 100 Mbps and 40 millisecond latency. Despite having such a fast LAN bandwidth, our protocol only utilizes a peak bandwidth of 1Gbps. 

All cryptographic operations are performed with computational security parameter $\kappa=128$ and statistical security $\lambda=40$. We consider set/table sizes of $n\in\{2^8, 2^{12}, 2^{16}, 2^{20}, 2^{24}\}$ and $n=2^{26}$ in some cases. Times are reported as the average of several trials.

\begin{figure*}[t!]\centering\scriptsize
% \begin{figure*}[h]\centering
\begin{tabular}{|l |l|| r | r |r |r|r||r | r |r |r|r||r|r|r|r|r|}
	\cline{3-17}
	\multicolumn{1}{c}{}          & \multicolumn{1}{c}{}                         & \multicolumn{5}{|c||}{LAN Time (sec.)}                   & \multicolumn{5}{|c|}{WAN Time (sec.)}                  & \multicolumn{5}{|c|}{Total Communication (MB)}            \\ \hline
	\multirow{2}{*}{Operation}    & \multirow{1}{*}{Protocol,}                   &                \multicolumn{5}{c||}{$n$}                 &                \multicolumn{5}{c|}{$n$}                &                 \multicolumn{5}{c|}{$n$}                  \\
	                              & \# Parties                                   & $2^8$   & $2^{12}$ & $2^{16}$ & $2^{20}$    &   $2^{24}$ & $2^8$ & $2^{12}$ & $2^{16}$ & $2^{20}$  &     $2^{24}$ & $2^8$ & $2^{12}$ & $2^{16}$ &    $2^{20}$ &      $2^{24}$ \\ \hline\hline
	\multirow{3}{*}{Intersection} & This                         \hfill ,3       & 0.02    & 0.03     & 0.2      & 4.9         &        117 &   2.3 & 2.5      & 6.4      & 41.4      &        902 &   0.2 &      3.0 &     48.1 &       769.4 &      12,318 \\ \cline{2-17}
	                              & \cite{CCS:KKRT16}                  \hfill ,2 & 0.2     & 0.2      & 0.4      & 3.8         &         58 &   0.6 & 0.6      & 1.3      & 7.5       &        106 &  0.04 &      0.5 &      8.1 &       127.2 &       1,955 \\ \cline{2-17}
	                              & \cite{ASIACCS:BlaAgu12}$^*$        \hfill ,3 & 2.9     & 23.4     & 374.4    & $^*$5,990.4 & $^*$95,846 &    -- & --       & --       & --        &           -- &    -- &       -- &       -- &          -- &            -- \\ \hline\hline
	\multirow{2}{*}{Joins/Union}  & This                        \hfill ,3        & 0.02    & 0.03     & 0.3      & 9.1         &        192 &   2.6 & 2.9      & 6.6      & 61.4      &      1,337 &   0.3 &      4.9 &     78.1 &     1,249.4 &      19,998 \\ \cline{2-17}
	                              & \cite{LTW13}$^*$                   \hfill ,3 & 2.0     & 8.0      & 128.0    & $^*$2,048.0 & $^*$32,768 &    -- & --       & --       & --        &           -- &    -- &       -- &       -- &          -- &            -- \\ \hline\hline
	\multirow{3}{*}{Cardinality}  & This                          \hfill ,3      & 0.01    & 0.02     & 0.2      & 3.1         &         74 &   1.1 & 1.1      & 1.8      & 15.8      &        267 &   0.1 &      2.0 &     32.6 &       521.5 &       8,344 \\ \cline{2-17}
	                              & \cite{PSWW18}a                     \hfill ,2 & $^*$0.1 & 2.2      & 9.1      & 86.6        &   $^*$1385 &    -- & 10.0     & 45.3     & 389.9     &  $^*$6,238 &    -- &     52.7 &    826.1 &     9,971.4 & $^*$159,542 \\ \cline{2-17}
	                              & \cite{PSWW18}b                      \hfill,2 & --      & --       & --       & --          &         -- &    -- & 13.0     & 56.2     & $^*$899.2 & $^*$14,387 &    -- &     14.3 &    171.3 & $^*$2,740.8 &  $^*$43,852 \\ \cline{2-17}
	                              & \cite{DBLP:conf/cans/CristofaroGT12}\hfill,2 & 1.0     & 16.0     & 262.0    & 4190.0      &     67,100 &    -- & --       & --       & --        &           -- &   0.1 &      0.4 &      6.2 &        99.0 &       1,584 \\ \hline\hline
	\multirow{2}{*}{Sum}          & This                           \hfill ,3     & 0.03    & 0.04     & 0.3      & 6.8         &        158 &   3.7 & 4.0      & 7.9      & 51.0      &      1,099 &   0.3 &      2.0 &     33.1 &       526.5 &       8,372 \\ \cline{2-17}
	                              & \cite{cryptoeprint:2017:738}    \hfill ,2    & 7.0     & 115.0    & 1,860.0  & 29,700.0    &    475,000 &    -- & --       & --       & --        &           -- &   0.1 &      1.9 &     30.2 &       483.0 &       7,728 \\ \hline
	Voter Intra-state             & This                          \hfill ,3      & 0.01    & 0.02     & 0.2      & 4.7         &        114 &   1.0 & 1.0      & 2.2      & 27.1      &        456 &   0.2 &      3.4 &     54.1 &       867.1 &      13,903 \\ \hline
	Voter Inter-state             & This                         \hfill ,3       & 0.01    & 0.02     & 0.3      & 7.0         &        134 &   1.6 & 1.6      & 4.0      & 45.4      &        747 &   0.4 &      5.7 &     91.3 &     1,463.9 &      23,482 \\ \hline
	Threat Log $N=2$              & This                         \hfill ,3       & 0.02    & 0.03     & 0.2      & 5.1         &        121 &   2.4 & 2.5      & 4.8      & 34.6      &        585 &   0.2 &      3.1 &     50.2 &       804.2 &      12,867 \\ \hline
	Threat Log $N=4$              & This                          \hfill ,3      & 0.05    & 0.09     & 0.9      & 17.9        &        388 &   6.6 & 6.8      & 13.1     & 108.7     &      1,739 &   0.6 &      9.7 &    155.4 &     2,487.8 &      39,804 \\ \hline
	Threat Log $N=8$              & This                          \hfill ,3      & 0.10    & 0.19     & 1.7      & 47.1        &      1,021 &  14.9 & 15.3     & 30.0     & 264.3     &      4,228 &   1.4 &     22.8 &    365.7 &     5,854.9 &      93,677 \\ \hline
\end{tabular}
\vspace{-0.3cm}
\caption{	\label{fig:compare}The running time in seconds and  communication overhead in MB for various join operations and application. The input tables each contain $n$ rows. The  \cite{PSWW18} protocol has two implementation where  \cite{PSWW18}b is optimized for the WAN setting. -- denotes that the running time is not available. * denotes that the running times were linearly extrapolated from the values of $n$ provided by the publication.}
\vspace{-0.2cm}
\end{figure*}



%
%\begin{figure*}[t!]\centering\footnotesize
%	% \begin{figure*}[h]\centering
%	\begin{tabular}{|l |l|| r | r |r |r|r|}
%		\cline{3-7}
%		\multicolumn{1}{c}{}              & \multicolumn{1}{c}{}                            & \multicolumn{5}{|c|}{Total Communication (MB)}    \\ \hline
%		\multirow{2}{*}{Operation}        & \multirow{1}{*}{Protocol,}                      &             \multicolumn{5}{c|}{$n$}              \\
%		                                  & \# Parties                                      & $2^8$ & $2^{12}$ & $2^{16}$ & $2^{20}$ & $2^{24}$ \\ \hline\hline
%		\multirow{2}{*}{Set Intersection} & This                         \hfill ,3          & 0.2   & 3.0      & 48.1     & 769.4    & 12,318.2 \\ \cline{2-7}
%		                                  & \cite{CCS:KKRT16}                     \hfill ,2 & 0.04  & 0.5      & 8.1      & 127.2    &  1,955.2 \\ \hline\hline
%		\multirow{1}{*}{Joins/Union}      & This                        \hfill ,3           & 0.3   & 4.9      & 78.1     & 1,249.4  & 19,998.1 \\ \hline\hline
%		                                 % & \cite{LTW13}$^*$                      \hfill ,3 &       &          &          &          &          \\ \hline\hline
%		\multirow{3}{*}{Cardinality}      & This                          \hfill ,3         & 0.1   & 2.0      & 32.6     & 521.5    &  8,344.0 \\ \cline{2-7}
%		                                  & \cite{PSWW18}a                   \hfill ,2      & --    & 52.7     & 826.1    & 9,971.4  &       -- \\ \cline{2-7}
%		                                  & \cite{PSWW18}b                   \hfill ,2      & --    & 14.3     & 171.3    & --       &       -- \\ \cline{2-7}
%		                                  & \cite{DBLP:conf/cans/CristofaroGT12} \hfill ,2  & 0.1   & 0.4      & 6.2      & 99.0     &  1,584.0 \\ \hline\hline
%		\multirow{2}{*}{Sum}              & This                           \hfill ,3        & 0.3   & 2.0      & 33.1     & 526.5    &  8,372.0 \\ \cline{2-7}
%		                                  & \cite{cryptoeprint:2017:738}    \hfill ,2       & 0.1   & 1.9      & 30.2     & 483.0    &  7,728.0 \\ \hline\hline
%		Voter Intra-state                 & This                           \hfill ,3        & 0.2   & 3.4      & 54.1     & 867.1    & 13,903.6 \\ \hline
%		Voter Inter-state                 & This                           \hfill ,3        & 0.4   & 5.7      & 91.3     & 1,463.9  & 23,482.8 \\ \hline\hline
%		Threat Log $\sim$ 2               & This                           \hfill ,3        & 0.2   & 3.1      & 50.2     & 804.2    & 12,867.9 \\ \hline
%		Threat Log $\sim$ 4               & This                           \hfill ,3        & 0.6   & 9.7      & 155.4    & 2,487.8  & 39,804.4 \\ \hline
%		Threat Log $\sim$ 8               & This                           \hfill ,3        & 1.4   & 22.8     & 365.7    & 5,854.9  & 93,677.5 \\ \hline
%	\end{tabular}
%\vspace{-0.3cm}
%	\caption{\label{fig:comm}	The total communication overhead in MB for various join operations and applications. The input tables each contain $n$ rows. The  \cite{PSWW18} protocol has two implementation where  \cite{PSWW18}b is optimized for the WAN setting. -- denotes that the running time is not available. }
%	\vspace{-0.2cm}
%\end{figure*}
%



\paragraph{Set Intersection} We first consider set intersection. In this case the two tables of our protocol consist of a single column which is used as the join-key. We compare our protocol to \cite{CCS:KKRT16} which is a two party set intersection protocol where the input sets each are known in the clear to one of the parties and one party learns the intersection exactly. This is contrasted by our three party protocol where the input and output sets are secret shared between the parties. That is, our protocol is composable \& supports outsourced MPC while \cite{CCS:KKRT16} does not and can not be trivially modified to do so without a large overhead. Both our protocol and \cite{CCS:KKRT16} were benchmarked on the same hardware. We also compare to the three party protocol of \cite{ASIACCS:BlaAgu12} which is composable and was not benchmarked on the same hardware. Due to the code of the \cite{ASIACCS:BlaAgu12} protocol not being publicly available, we cite their benchmarks which were performed on three AMD Opteron computers at 2.6GHz connected on a 1Gbps LAN network. Given the relative performance of our machines, we believe this to yield a fair comparison. This protocol first sorts the two input sets/tables which in practice requires $O(n\log^2 n)$ overhead\cite{ASIACCS:BlaAgu12}. In contrast, our protocol and \cite{CCS:KKRT16} has $O(n)$ overhead and $O(1)$ rounds. 

This asymptotic difference also translates to a large difference in the concrete running time as shown in \figureref{fig:compare}. Out of these three protocol \cite{CCS:KKRT16} is the fastest requiring 3.8 seconds in the LAN setting to intersect two sets of size $n=2^{20}$ while our protocol requires 4.9 seconds. However, our protocol is fully composable while \cite{CCS:KKRT16} is not. Considering this we argue that a slowdown of $1.28\times$ is acceptable. When compared to \cite{ASIACCS:BlaAgu12} which provides the same composable functionality, our protocol is estimated\footnote{We linearly extrapolate the overhead of their protocol, despite having $O(n\log n)$ complexity.} to be $1220\times$ faster.

In the WAN setting our protocol has a relative slowdown compared to \cite{CCS:KKRT16}. This can be contributed to our protocol requiring more rounds and communication. For instance, with $n=2^{20}$ the protocol of \cite{CCS:KKRT16} in the WAN setting requires 7.5 seconds while our protocol requires 41 seconds, a difference of $5.5\times$. With respect to the communication overhead, our protocol for $n=2^{20}$ requires 769 MB of communication and \cite{CCS:KKRT16} requires 127 MB, a difference of $6\times$. The WAN running time and communication overhead of \cite{ASIACCS:BlaAgu12} is not known due to their code not being publicly available.


\paragraph{Joins/Union} 
The second point of comparison is performing an inner join protocol on two tables consisting of five columns of 32-bit values. We note that \cite{ASIACCS:BlaAgu12} is capable of this task but no performance results were available. Instead we compare with the join protocol of \cite{LTW13}. This protocol is composable but requires that the cardinality of the intersection be revealed after each join is performed. As previously discussed, this leakage limits the suitability of the protocol in many applications. The numbers reported for \cite{LTW13} are from their paper and the experiments were performed on three servers each with 12 CPUs at 3GHz in the LAN setting. As can be seen in \figureref{fig:compare}, we estimate\footnote{Again, we linearly extrapolate the overhead of the protocol.} our protocol is roughly $200\times$ faster in the LAN setting. For example, with $n=2^{20}$ our protocol requires a running time of 9.1 seconds while \cite{LTW13} requires a running time of 2048 seconds. Moreover, our protocol scales quite well with the addition of these extra four columns as compared to a intersection protocol. For example, in the WAN setting an intersection with $n=2^{20}$ requires 41 seconds while the addition of the four columns results in a  running time of 61 seconds. For both protocols, operations such as left join and unions can be performed with little to no additional computation as compared to inner join.

We observed the following relative performance of the various operations of our protocol. Secret sharing the input tables tool 3\% of the time, computing the randomized encodings via \proto{encode} required 50\%, constructing the cuckoo hash table via \proto{Permute} required 6\%, selecting the rows from the cuckoo table required 26\%, and the final circuit computation via \f{mpc} required 14\%. These percentages were obtained for $n=2^{20}$ in the LAN setting and hold relatively stable regardless of $n$.



\paragraph{Cardinality} The set cardinality protocol presented here also outperforms all previous protocols. As described in \sectionref{sec:card}, our cardinality protocol allows the omission of the switching network which reduces the amount of communication and overall running time. We demonstrate the performance by comparing with the two-party protocols of Pinkas et al. \cite{PSWW18} and De Cristofaro et al. \cite{DBLP:conf/cans/CristofaroGT12}. The protocol of  \cite{PSWW18} was benchmarked on two multi-core i7 machines at 3.7GHz and 16GB of RAM with similar network settings. For the protocol of \cite{DBLP:conf/cans/CristofaroGT12}, we performed rough estimates on the time required for our machine to perform the computation without any communication overhead. For sets of size $n=2^{20}$ our protocol requires 3.1 seconds in the LAN setting and 15.8 in the WAN setting. The next fastest protocol is \cite{PSWW18} which requires 86.6 seconds in the LAN setting and 390 seconds in the WAN setting. In both cases this represents more than a $20\times$ difference in running time.  \cite{PSWW18} considers a variant of their protocol optimized for the WAN setting which reduces their communication at the expense of increased running time. The protocol of \cite{DBLP:conf/cans/CristofaroGT12} requires the most running time by a large margin due to the protocol being based on exponentiation. Just to locally perform these public key operations requires roughly 4200 seconds of computation on our benchmark machine, a difference of $1350\times$. However, the protocol of \cite{DBLP:conf/cans/CristofaroGT12} also requires the least amount of communication, consisting of 99MB for $n=2^{20}$ while our protocol requires 521MB followed by \cite{PSWW18}  with almost 10GB.

\paragraph{Sum} The last generic comparison we perform is for securely computing the weighted sum of the intersection. Our protocol for performing this task is described in \sectionref{sec:card}. We compare to the protocol of Ion et al. \cite{cryptoeprint:2017:738} which is the protocol behind Google's Join-and-Compute. This protocol can be viewed as an extension of the public key based cardinality protocol of \cite{DBLP:conf/cans/CristofaroGT12}. In particular, \cite{cryptoeprint:2017:738}  also revealed the cardinality of the intersection and then performs a secondary computation using Paillier homomorphic encryption to compute the sum. Although this protocol reveals more information than ours, we still think it is a valuable point of comparison. Not surprisingly,  the protocol of \cite{cryptoeprint:2017:738} requires significantly more computation time than our protocol. For a dataset size of $n=2^{20}$, their protocol requires almost 30000 seconds to just perform the public key operations without any communication. Our protocol requires just 6.8 seconds in the LAN setting and 51 in the WAN setting. Both of these protocols also consume roughly the same amount of communication with \cite{cryptoeprint:2017:738} requiring 483 MB and our protocol requires 527 MB, a increase of just 9 percent. 

\paragraph{Voter Registration} We now turn our attention to the application of auditing the voter registration data between and within the states of the United States as described in 
\iffullversion
\sectionref{sec:voter}.
\else
\sectionref{sec:voter} \& \appendixref{sec:voterDetails}. 
\fi
In summary, this application checks that a registered voter is not registered in more than one state and cross validates that their current address is correct. Only the identities of the voters which have conflicting data are revealed to the appropriate state to facilitate a process to contact the individual. In addition, the application can be extended to assist the process of enrolling unregistered citizens. This audit process is performed using two types of join queries. First, each state computes a left join between the DMV database and the list of registered voters. In \figureref{fig:app} we call this join \emph{Voter Intra-state}. For all pairs of states, these tables are then joined to identify any registration error, e.g. double registered. This join is referred to as \emph{Voter Inter-state}. Performance metrics are reported for each of these joins individually and then we estimate the total cost to perform the computation nation wide. 

As shown in \figureref{fig:app}, our  protocol can perform the \emph{Voter Intra-state} join with an input set size of 16 million voters ($n=2^{24}$) in 115 seconds on a LAN network and in 456 seconds on a WAN network. Considering all but three states have a voting population less than $n=2^{24}$, we consider this a realistic estimate on the running time overhead. Our protocol also achieves relatively good communication overhead of 13.9 GB, where each of the servers sends roughly one third of this. On average, that is 830 bytes for each of the $n$ records. Given these tables, the \emph{Voter Inter-state} join is performed between all pairs of states. For two states with $n=2^{24}$, the benchmark machine required 135 seconds in the LAN setting and 748 seconds in the WAN setting. The added overhead in this second join protocol is an additional \texttt{where} clause which requires a moderate sized binary circuit to be securely evaluated. This join requires 23.4 GB of communication.

Given the high value and low frequency of this computation we argue that these computational overheads are very reasonable. Given the current population estimates of each state, we extrapolate that the overall running time to run the protocol between all pairs of 50 states in a LAN setting would be 53,340 seconds (14.8 hours) or  285,687 seconds (about 80 hours) in the WAN setting. However, the running time in the WAN setting could easily be reduced by running protocols in parallel and increasing the bandwidth above the relatively low 100Mbps per party. The total communication overhead is 9,131 GB which is the main bottleneck.  While this amount of communication is non-negligible, the actual dollar amount on a cloud such as AWS\cite{aws} is relatively low (given the importance of the computation), totaling roughly \$820\cite{aws_pricing}. 


\paragraph{Threat Log} In this application $N$ party secret share their data between the three computational parties and delegate the task of identifying the events that appear in at least $k$ out of the $N$ data sets. As described in \sectionref{sec:threatlog}, the protocol proceeds by taking the union of the sets and then the number of times each event occurred is counted and compared against $k$. Each event that appears more than $k$ times is then revealed to all parties. The union protocol can only function with respect to two input sets. To compute the union of $N$ sets we use a binary tree structure where pairs of sets are combined. As such, there are a total of $N-1$ union operations and a depth of $\log N$ protocol instances.

When benchmarking we consider $N=\{2,4,8\}$ input sets each of size $n\in \{2^8, 2^{12}, 2^{16}, 2^{20}, 2^{24}\}$. Since we do not reveal the size of the union, the final table will be of size $nN$. For $N=2$ sets each with $n=2^{24}$ items our protocol requires 121 seconds in the LAN setting and 586 seconds in the WAN setting. The total communication is 804MB, or approximately 24 bytes per record. If we increase the number of sets to $N=8$ we observe that the LAN running time increases to 1,021 seconds and 4,228 seconds  in the WAN setting. Given the the total input size increased by $4\times$, we observe roughly an $8\times$ increase in running time. This difference is due to each successive union operation being twice as big. Theoretically the running time and communication of this protocol is $O(nN\log N)$.


\input{comparison}

\bibliographystyle{alpha}
\bibliography{bib,cryptobib/crypto}

\end{document}
