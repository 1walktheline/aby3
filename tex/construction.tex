\section{Our Construction}



\subsection{Overview}

Our core protocol is a technique for obliviously mapping together records that have equal keys. In particular, for each secret shared row $\share{X[i]}$ our protocol obliviously maps the row $\share{Y[j]}$ to the $i$ position of a new table $\share{Y'}$ if the \emph{joined on} columns compare for equality. In the event that no such $Y[j]$ exists then an arbitrary $j$ is used instead. Once the mapping is performed the output table can be constructed by an MPC protocol\cite{aby3} that compares the $i$th row of $\share X$ and $\share{Y'}$. 

Without loss of generality let us assume that the columns $X_0$ and $Y_0$ are being joined on. Our protocol begins by generating a \emph{randomized encoding} for all of the secret shared elements $\share x\in \share{X_0}, \share y\in \share{Y_0}$. The ideal functionality for this encoding is to take secret shares from the parties, apply a PRF $F_k$ to the reconstructed value using a internally sampled key $k$, and return the resulting value to one of the three parties. For $\share x\in \share{X_0}$, party 0 will learn $F_k(x)$ while party 1 will learn $F_k(y)$ for $\share y\in \share{Y_0}$.

Party 1 proceeds by constructing a \emph{secret shared} cuckoo hash table $\share{\hat Y}$ for the rows of $\share{Y}$ where the hash function values employed are defined as $h_i(y) = H( i || F_k(y))$. That is, party 1 knows the hash function values but the contents of the hash table remains secret shared between the parties. To prevent the other parties from learning information about the randomized encodings $F_k(y)$, party 1 must obliviously permute their shares to the desired position of the cuckoo hash table. We achieve this using a three party oblivious permutation protocol which further randomizes the secret shares.

It is now the case that $\share{\hat Y}$ is a valid cuckoo hash table of $\share Y$ which is in a secret shared format. Party 0 who knows the randomized encodings $F_k(x)$ for all $\share x\in \share{X_0}$ now must query $\share{\hat Y}$ at the slots indexed by $h_i(x)= H( i || F_k(x))$ and compare this with the corresponding row of $\share X$. In particular, assuming we use two hash function, then party 0 constructs an \emph{oblivious switching network} that maps the shares $\share{\hat Y_{h_0(x)}}$ and $\share{\hat Y_{h_1(x)}}$ to be aligned with $\share x$.

Once the shares  $\share{\hat Y_{h_0(x)}}, \share{\hat Y_{h_1(x)}}$ and $\share x$ are aligned, the parties employ an MPC protocol to compare the joined on columns to compute a secret shared bit denoting whether $x$ matches with one of the rows. When computing an inner join query, only rows of $X$ where the comparison bit is set to one are considered valid while the other rows are set to \texttt{NULL}. Note that when columns of $Y$ are selected the corresponding values are obtained from $\share{\hat Y_{h_i(x)}}$ for the $i$ that the comparison succeeded on. Left joins work in a similar way except that all rows of $X$ are included while the comparison bit is used to select the columns of $Y$ or set the fields to \texttt{NULL}. Finally, unions can be computed by including all of $Y$ in the output and all of the rows of $X$ where the comparison bit is set to zero. Regardless of the type of join, the protocols do not reveal any information about the set. In particular, not even the cardinality of the join is revealed.

\subsection{Randomized Encodings}

Randomized encodings enable the parties to coordinator their secret shares without revealing the underlying values. Crucially, the equality  of two randomized encodings implies the equality of the encoded value but nothing more. The ideal functionality of the encoding process is given in \figureref{fig:randomized-encode-ideal}. It considers two commands which allow the parties to initialize the internally stored key $k$ and later generate encoding under that key. In particular, the parties are allowed to send secret shares of a value $x$  to the ideal functionality and destinate which party should learn the encoding. This functionality has several interesting properties. First, all encoding that have not been observed are uniformly distributed in that parties view. Secondly, learning an encoding $F_k(x)$ does not reveal any information about $x$ beyond being about to compare it for equality with other encodings. 

\begin{figure}[ht]
	\framebox{\begin{minipage}{0.95\linewidth}
			Parameters: $N$ parties denoted as party 0 through party N-1. The input domain $\{0,1\}^\sigma$ and output domain $\{0,1\}^\ell$ for a pseudorandom function $F$.
			
			\begin{enumerate}
				\item[] [Key Gen] Upon receiving command $(\textsc{KeyGen})$ from all parties, sample a uniformly random key $k$ from the key space of $F$ and store it internally.
				
				\item[] [Encode] Upon receiving command $(\textsc{Encode}, \share x, i)$ from all parties, if $k$ is uninitialized compute $F_k(x)$ and send it to party $i$. 
			\end{enumerate}
	\end{minipage}}
	\caption{The Randomized Encoding ideal functionality \f{encode}}
	\label{fig:randomized-encode-ideal}	
\end{figure}


\paragraph{LowMC Encodings.}
We consider two protocol for realizing \f{encode}. The first is optimized with respect to computational overhead and is based the LowMC circuit\cite{lowmc} which computes a block cipher. With this approach we use the framework of \cite{aby3, highthroughput} to evaluate the a circuit implementing this PRF and reveal the output to the designated party. When coupled with these honest majority MPC protocols, this approach results in extremely high throughput. For instance, our implementation can compute one million encodings in a few seconds.

LowMC is a family of MPC optimized block ciphers based on a binary substitution-permutation network. The cipher is parameterized by a block size $n$, keys size $\kappa$, s-boxes per layer $m$ and the desired data complexity $d$. Given the desired security level, e.g. 128 bits, the required number of rounds $r$ can then be computed as a function of these parameters. The structure of the cipher is specifically optimized to reduced the number of s-boxes (\textsc{and} gate) and the number of rounds. For each of the $r$ rounds, the cipher adds part of the key to the current state, multiplies it with a public binary matrix and then applies 3 bit s-boxes in parallel to a subset of the state. With respect to performance metrics, the most costly operation is the application of the s-boxes and the number of rounds required. 

An important observation of our protocol is that the adversary only sees a bounded amount of block cipher output. In particular, the number of blocks observed $d$ is exactly the size of the table which is being encoded. For our implementation we set $d=2^{30}$ and optimize the remaining parameters. Our second observation is that our protocol uses the cipher as a PRF and does not require a excessive number of output bits. The desired property of the encodings is that the probability of spurious collisions between encodings is bounded by the statistical security parameter $\lambda$. Given table sizes of $|X|$ and $|Y|$, this probability is bounded by $n-\log |X|-\log |Y|\geq \lambda$. Considering the standard of setting $\lambda=40$, we observe that $n=80$ gives a sufficient margin for realistic table sizes. The remaining parameter $m$ was optimized empirically and set to be $m=14$ which resulting in $r=13$. This results in the evaluation of the LowMC requiring 13 rounds of communication and a total of 546 \textsc{and} gates (bits of communication).



\paragraph{Diffie-Hellman Encodings.} Our second approach uses a Diffie-Hellman styled assumption to construct a PRF on shared input. In particular, for the share $\share x$ the protocol computes $F_k(x) = g^{xk}$. This approach has the advantage of not computing a PRF circuit within MPC which can be costly in some settings.


\todo{talk about how to convert the shares.  We have binary shares and we need mod $p$ shares... }


%\paragraph{Encodings Long Elements.}\todo{If the element is too large to be encoded above, we can choose a random binary matrix and (locally) multiply the shares with it. We then encode. Given that this matrix is chosen after the shares, everything should be good.}

\subsection{Oblivious Switching Network}

A switching network was introduced by Mohassel and Sadeghian\cite{MS13} as a circuit which can obliviously transform a vector $A=\{A_1,...,A_n\}$ such that the output is $A'=\{A_{\pi(1)}, ..., A_{\pi(m)}\}$ for an arbitrary function $\pi : [m]\rightarrow[n]$. The protocol of \cite{MS13} was designed in the two party setting where the first party inputs $A$ while the second party inputs a description of $\pi$. More recently Carmer et al. \cite{CMRS18} generalized the construction and demonstrated how this technique can be utilized for private set intersection. In both cases these switching networks require  $O(n\log n)$ cryptographic operations. Building on this general paradigm, we introduce a new oblivious switching network protocol tailored for the honest majority setting which significantly improves the efficiency. In particular, our protocol has linear overhead and does not require the use of oblivious transfer. 

The ideal functionality of our protocol is given in \figureref{fig:perm-ideal}. This functionality considers three parties, a \emph{programer}, a \emph{sender} and a \emph{receiver}. The programmer has a description of an arbitrary function  $\pi:[m]\rightarrow[n]$ while the sender a vector $A$ containing $n$ elements each consisting of $\sigma$ bits. At the completion of the protocol, the programmer and the receiver should hold a 2-out-of-2 secret sharing of $A'=\{A_{\pi(1)}, ..., A_{\pi(m)}\}$.


The protocols described below assume the vector being transformed is the private input of the sender. However, our larger join protocols require the transformations to be applied to secret shared vectors. In particular, parties 0 and 1 both hold secret shares of $A$. This is achieved by using the oblivious switching protocol to transform the shares of the sender. The programmer who knows the program $\pi$ can then locally permute their local share and combine this with the output of the oblivious switching protocol.

\begin{figure}[ht]
	\framebox{\begin{minipage}{0.95\linewidth}
			Parameters: $3$ parties denoted as the \emph{programmer}, \emph{sender} and \emph{receiver}. Elements are strings in $\{0,1\}^\sigma$. An input vector size of $n$ and output size of $m$.
			
			\begin{enumerate}
				\item[] [Switch] Upon the command $(\textsc{switch}, \pi)$ from the \emph{programmer} and $(\textsc{switch}, A)$ from the \emph{sender}, the functionality performs:
				\begin{enumerate}
					\item Interpret $\pi: [m]\rightarrow [n]$ as a function and $A\in \{0,1\}^{n\times \sigma}$ as a vector of $n$ elements. 
					\item Uniformly sample two $m$ elements vector $B^0, B^1\gets \{0,1\}^{m\times \sigma}$ such that for all $i\in [m], A_{\pi(i)} = B^0_i \oplus B^1_i$.
					\item Send $B^0$ to the \emph{programmer} and $B^1$ to the \emph{receiver}.
				\end{enumerate}
			\end{enumerate}
	\end{minipage}}
	\caption{The Oblivious Switching Network ideal functionality \f{switch}}
	\label{fig:perm-ideal}	
\end{figure}


\paragraph{Permutation Network}

We begin with a restricted class of switching networks where the programming function $\pi$ is injective. These types of programs do not allow a single input element $A_i$ to be mapped to more than one location in the output vector. 

Our protocol begins by having the programmer sampling a two random functions $\pi_0,\pi_1$ such that $\pi_1 \circ \pi_0 = \pi$, $\pi_0$ is bijective and $\pi_1$ is injective. The programmer sends $\pi_0$ and a random vector $S$ of $n$ elements to the sender who send $T := \{V_{\pi(1)} \oplus S_0, ...,V_{\pi(n)} \oplus S_n \}$ to the receiver. The programmer sends $\pi_1$ to the receiver. The final shares of the permuted $V$ are defined as the programmer holding $\{S_{\pi_1(1)}, ..., S_{\pi_1(m)}\}$ and the receiver holding $\{T_{\pi_1(1)}, ..., T_{\pi_1(m)}\}$.

The simulatability of this protocol is straight forward. The view of the sender contains a uniformly distributed permutation of $n$ elements $\pi_0$ and a uniformly distributed vector $S$. Similarly, the view of the receiver contains $\pi_1: [m]\rightarrow [n]$ which is uniformly distributed (when $\pi_0$ is unobserved) and the vector $T$ is uniformly distributed given that it is masked by $S$. One important observation of this simulation is that $\pi_0,S$ can be generated locally by parties 0 and 1 using a common source of randomness, e.g. a seeded PRG. This reduces the rounds to 1 and linear communication complexity. 
s
\paragraph{Universal Switching Network}

A universal switching network with an \emph{arbitrary} program $\pi : [m]\rightarrow [n]$ can be constructed in three phases\cite{MS13, CMRS18}. In particular, the input vector $A$ will have three transformation applied $A\overset{\pi_1}{\rightarrow}B\overset{\pi_2}{\rightarrow}C\overset{\pi_3}{\rightarrow}D=\pi(A)$.
\begin{enumerate}
	\item $B:=\pi_1(A)$:  The input vector $A$ is permuted by the injective function $\pi_1:[m]\rightarrow[n]$ such that if $\pi$ maps an input position $i$ to $k$ outputs positions (i.e. $k=|preimage(\pi,i)|=|\{ j : \pi(j)=i \}|$), then there exists a $j$ such that $\pi_1(j)=i$  and $\{\pi_1(j)+ 1,...,\pi_1(j )+k \} \cap image(\pi) = \emptyset$. That is, wherever position $i$ is mapped by $\pi_1$, it should be followed by $k-1$ input positions that do not appear in the final output. The parties then use a permutation network to compute $B:=\pi_1(A)$.
	
	\item $C:=\pi_2(B)$: The intermediate vector $B$ is transformed by a duplication network $\pi_2:[m]\rightarrow[m]$ such that if position $A_i$ is mapped to $k$ positions in $\pi(A)$, then $\{ C_{j},...,C_{j+k}\} = \{A_i\}$ where $\pi_1(j)=i$. That is, $C$ takes $B$ and copies $B_{j}$ into the next $k-1$ positions. 
	
	\item $D:=\pi_3(C)$: The final transformation $\pi_3:[m]\rightarrow[m]$  permutes $C$ to have the same ordering as $\pi(A)$. That is, the elements $\{ C_{j},...,C_{j+k}\}$ which all have the value  $A_i$ are arbitrary mapped to the $k$ positions $\{ j : \pi(j)=i \}$.
\end{enumerate}
Observer that steps 1 and 3 can both be implemented using the oblivious permutation protocol.% However, note that our oblivious permutation functionality is defined for $m\leq n$ while the switching network has no such restriction. This can be overcome by artificially padding the input vector $V$ with dummy items to be of size $\max(m, n)$. 

What remains is how to efficiently implement the duplication network $\pi_2:[m]\rightarrow[m]$. This transformation can be characterized by a bit vector $b$ of length $m-1$ where the $i$th bit denotes whether the item at position $i$ should have the same value as position $i+1$. This observation gives rise to a natural protocol: for $i\in [m-1]$, if $b_i=1$ then use MPC to copy $B_i$ into $B_{i+1}$. The primary challenge is to achieve this while using a constant number of communication rounds which prevents the use of a generic MPC protocol such as \cite{aby3, highthroughput}.

To obliviously select $B_i$ or $B_{i+1}$ conditioned on the programming bit $b_i$ we require the resulting value $C_{i+1}$ be secret shared. In particular, we consider the setting there the programmer knowns the programming bit $b_i$ while $B_i$ and $B_{i+1}$ are private input of the sender. At the end the programmer and sender will hold secret share $C^0_{i+1}, C_{i+1}^1$ of $C_{i+1}:=b_i B_i + (1-b_1)B_{i+1}$. The sender begins by sampling three random strings $C_{i+1}^1, w_0,w_1\gets \{0,1\}^\sigma$ and a random bit $\phi\gets \{0,1\}$. They construct two messages $m_0=B_{i+1}^1\oplus C_{i+1}^1\oplus w_\phi$ and $m_1= B_i\oplus C_{i+1}^1 \oplus w_{\phi\oplus 1}$. The sender sends $w_0,w_1$ to the receiver and sends $m_0,m_1,\phi$ to the programmer who sends $\rho=\phi\oplus b_i$ to the receiver. The final share are constructed by having the receiver send $w_\rho$ to the programmer who computes $C_{i+1}^0:=m_{b_i}\oplus w_{\rho}$.

The simulation of this protocol has two key parts. First, party 0 learns only one of the keys $w_0,w_1$ which determines which of the secret share $(B_{i+1}\oplus C_{i+1}^1$ or $  B_i\oplus C_{i+1}^1)$ they can one-time-pad decrypt. As such, the other share is uniformly distributed in their view. Similarly, the bit $\rho$ that the programmer sends to the receiver is uniformly distributed given that $\phi$ is not contained in the view of the receiver . The remaining messages $w_0,w_1,\phi$ which are uniformly sampled are trivial to simulate. Moreover, sending these messages can be optimized away when a PRG seed is pre-share between the appropriate parties.

The protocol just described considers the setting where the messages $B_i,B_{i+1}$ are the private input of the receiver. However, we require that at each iteration the messages being selected is either $C_i$ or $B_{i+1}$ where the former was computed in the previous iteration and is secret shared between the programmer and sender. Fortunately, a trivial modification yields the desired functionality. The sender simply utilizes their share of $C_i$ instead if $B_i$ while the programmer can now compute $C_{i+1}^0:=m_{b_i}\oplus w_{\rho}\oplus b_iC_{i}^0$. It is now the case that both shares of $C_i$ are obliviously multiplied by $b_i$. \figureref{fig:switching-net} provides a formal description of the full switching network protocol.

\begin{figure}[ht]
	\framebox{\begin{minipage}{0.95\linewidth}\small
			Parameters: $3$ parties denoted as \emph{programmer}, \emph{sender} and \emph{receiver}. Elements are strings in $\{0,1\}^\sigma$. An input vector size of $n$ and output size of $m$.
			
			\begin{enumerate}
				\item[] [Permute] Upon the command $(\textsc{Permute}, \pi)$ from the \emph{programmer} and $(\textsc{Permute}, A)$ from the \emph{sender}.  $\pi: [m]\rightarrow [n]$ is parsed as a \emph{injective} function and  $A\in \{0,1\}^{n\times \sigma}$ as a vector of $n$ elements. Then:
				\begin{enumerate}
					\item The \emph{programmer} samples a uniformly random bijective function $\pi_0 : [n]\rightarrow[n]$ and computes the injective function $\pi_1 :[n] \rightarrow[m]$ such that $\pi_1\circ \pi_0 = \pi$.  $\pi_0 $ and a random vector $S\gets \{0,1\}^{n\times \sigma}$ are sent to the \emph{sender}.
					\item The \emph{sender} computes and sends $B := \{ A_{\pi_0(1)} \oplus S_1, ..., A_{\pi_0(n)} \oplus S_n\}$ to the \emph{receiver}.
					\item The \emph{programmer} sends $\pi_1$ and a random vector $T\gets\{0,1\}^{m\times\sigma}$ to the \emph{receiver} who outputs $C^0:=\{B_{\pi_1(1)} \oplus T_1,...,B_{\pi_1(m)}\oplus T_m\}$. The \emph{programmer} outputs $C^1:=\{ S_{\pi_1(1)}\oplus T_1,...,S_{\pi_1(m)}\oplus T_m\}$.
				\end{enumerate}
				
				\item[] [Switch] Upon the command $(\textsc{Switch}, \pi)$ from the \emph{programmer} and $(\textsc{Switch}, A)$ from the \emph{sender}. $\pi: [m]\rightarrow [n]$ is parsed as a function and  $A\in \{0,1\}^{n\times \sigma}$ as a vector of $n$ elements. Then:
				\begin{enumerate}
					\item If $n<m$, the \emph{sender} redefines $A$ to be $A := A || \{0\}^{(m-n)\times \sigma}$ and all parties redefine $n:=m$.
					\item The \emph{programmer} samples an injective function $\pi_1:[m]\rightarrow [n]$ such that for $i\in image(\pi)$ and $k=|preimage(\pi, i)|$, there exists a $j$ where $\pi_1(j)=i$ and $\{\pi_1(j+1), ...,\pi_1(j+k) \}\cap image(\pi)=\emptyset$.
					
					The \emph{programmer}  sends $(\textsc{Permute}, \pi_1)$ to \proto{switch} and the \emph{sender} sends $(\textsc{Permute}, A)$. The \emph{programmer} receives $B^{0}\in \{0,1\}^{m\times \sigma}$ in response and the \emph{receiver} receives $B^{1}\in \{0,1\}^{m\times \sigma}$. 
					
					\item The \emph{programmer}  computes the vector $b\in\{0,1\}^{m}$ such that for $i\in image(\pi)$ and $k=|preimage(\pi, i)|$, $b_j = 0$ and $b_{j+1}=...=b_{j+k}=1$ where $\pi_1(j)=i$.
					
					The \emph{receiver} samples three $m$ element vectors $C^{1}, W^0,W^1\gets \{0,1\}^{m\times \sigma}$ and $\phi\gets\{0,1\}^m$. They set $C^{1}_1:=B^{1}_1$ and computes 
					\begin{align*}
						M^0_i&:= B^1_{i}\ \ \, \oplus C^{1}_i \oplus W^{\phi_i}_i\\
						M^1_i&:= C^1_{i-1} \oplus C^{1}_i \oplus W^{\phi_i\oplus 1}_i
					\end{align*}
					for $i\in \{2,...,m\}$. The \emph{receiver} sends $M,\phi$ to the \emph{programmer} and $C^{1},W$ to the \emph{sender}. The \emph{programmer} sends $\rho:=\phi\oplus b$ to the \emph{sender} who responds with $\{ W^{\rho_i}_i : i\in [m] \}$. The \emph{programmer} defines $C^{0}_1:=B^{0}_1$ and computes 
					$$
						C^{0}_i:= M^{b_i}_i \oplus W^{\rho_i}_i\oplus b_iC^{0}_{i-1}
					$$
					for $i\in \{2,...,m\}$.
					\item The \emph{programmer} computes the permutation $\pi_3$ such that for  $i\in image(\pi)$ and $k=|preimage(\pi, i)|$, $\{\pi_3(\ell) : \ell\in preimage(\pi, i)\}=\{j, ..., j +k\}$ where $i=\pi_1(j)$.	The \emph{programmer} sends $(\textsc{Permute}, \pi_3)$ to \proto{switch} and the \emph{sender} sends $(\textsc{Permute}, C^{1})$.  The \emph{programmer} receives $S\in \{0,1\}^{m\times \sigma }$ in response. The \emph{receiver} receives and outputs $D^{1}\in \{0,1\}^{m\times \sigma }$.
					
					The \emph{programmer} outputs $D^{0}_i:=S_i\oplus C^{0}_{\pi_3(i)}$ for $i\in [m]$.
				\end{enumerate}
			\end{enumerate}
	\end{minipage}}
	\caption{The Oblivious Switching Network protocol \proto{switch}. }
	\label{fig:switching-net}	
\end{figure}





\subsection{Join Protocols}

The full join protocol can be constructed using the presented building blocks. When joining two tables $X,Y$ our protocol can be divided into four phases:

\begin{enumerate}
	\item Compute randomized encodings of the joined on column(s). 
	\item Party 1 constructs a cuckoo table for table $Y$ and arranges the secret shares using an oblivious permutation protocol. 
	\item For each row $x$ in $X$, party 0 uses an oblivious switching network maps the corresponding location $i_1,i_2$ of the cuckoo hash table to a secret shared tuple $(x, y_{i_1}, y_{i_2})$.
	\item The joined on column(s) of $x$ is then compared to that of $y_{i_1}, y_{i_2}$. If there is a match the output values are constructed. Otherwise the output row is set to \texttt{NULL}.
\end{enumerate} 

\paragraph{Randomized Encodings}
We begin by generating randomized encodings of the columns being joined on. For example, 
$$
	\texttt{select }* \texttt{ from } X \texttt{ inner join } Y \texttt{ on } X_1 = Y_1 \texttt{ and } X_2 = Y_3
$$
In this case there are two joined on columns, $X_1,X_2$ from $X$ and $Y_1,Y_3$ from $Y$. Our protocol requires that the parties generate a randomized encoding for each row of $X$ and $Y$. That is, the parties must send a secret share of $\share{X_1[i] || X_2[i]}$ and $\share{Y_1[i] || Y_3[i]}$ to \f{encode} where $||$ denotes concatenation. 

There are two major challenges to efficiently implement this functionality in a composable setting. First, the encoding functionality takes as input a secret shared value consisting of $\sigma$ bits. Moreover, when implemented using the LowMC block cipher optimal performance is achieved with a block size of $\sigma=80$ bits. However, the number of bits being encoded is dependent on the column sizes and which columns are being joined on. Instead of changing the functionality of the encoding procedure we show that a preprocessing step that reduced the size of the inputs without degrading the security properties of the encodings. 

Specifically, once the tables being joins on have been specified, the parties execute a coin flipping protocol to jointly pick a random string $s\in \{0,1\}^\kappa$. Given $s$ the parties deterministically generate a random matrix $E\gets\{0,1\}^{m\times \sigma}$. The parties can then locally compute $\share{E_x}=\share{X_1[i] || X_2[i]} E$ and $\share{E_y}=\share{Y_1[i] || Y_3[i]} E$ where $m$ denotes the total bit length of the columns being joined on. These secret shares can then be forwarded to the \f{encode}

To preserve correctness and security we require the probability of the following game outputting 1 be negligible in $\lambda$. 
\begin{quote}
Have the adversary select a $X\subset \{0,1\}^{m}$ of size $poly(\lambda)$ and then uniformly at random sample $E\gets \{0,1\}^{m\times \sigma}$. If there exists distinct $x_1,x_2\in X$ such that $x_1E = x_2E$, output 1, otherwise 0. 
\end{quote}

\todo{Show that this holds and that $\sigma=80$ is chill for $\lambda = 40$.}

The second challenge we face has to do with the composibility of our protocols. Namely, after a previous join operation, some (or all) of the rows being joined can be invalid. We require that the randomized encoding does not reveal which rows are invalid. This is achieved by secret sharing a bit for each row which encodes whether or not the row is currently valid. For tables that are input by a party, these bits are publicly set to 1. Given this bit \share{b}, the parties can generate a random $\sigma$ bit share $\share{r}$ for each value $\share{x}$ that is encoded. Before $\share{xE}$ is sent to \f{encode}, the parties compute $\share{x'}:=\share{xE}\oplus \share{\overline{b}}\share{r}$ and send $\share{x'}$ instead. In the event that the current row is valid, this alteration has no effect on the computation due to \share{\overline{b}}\share{r} being a secret sharing of zero. However, when the row was invalid the value of $x'$ is uniformly distributed. Conditioned on $x'$ not colliding with another value being encoded, the resulting encoding is uniformly distributed and unique with overwhelming probability. 

\todo{Pr. of collision.}

Putting everything together, the parties will jointly sample a random binary matrix $E\gets\{0,1\}^{m\times \sigma}$ if $m>\sigma$ and $E=I$ otherwise. For each set of join on columns $\share{Z_{i_1}},...\share{Z_{i_l}}$ that are to be encoded, they parties jointly sample a secret shared value $\share{r}$ where $r\in\{0,1\}^\sigma$ and compute $\share{z'}:=\share{Z_{i_1}||...||Z_{i_l}}E \oplus \share{\overline{b}}\share{r}$. \share{z'} is sent to \f{encode} who returns the encoding for that row to the appropriate party. In particular, party 0 learns the randomized encodings $\mathbb{E}_x$ for the joined on columns of the left table $X$ and party 1 learns the encodings $\mathbb{E}_y$ for the right table $Y$.

\paragraph{Constructing the Cuckoo Table}

The next phase of the protocol is for party 1 to construct a secret shared cuckoo table for $Y$ where each row is inserted based on its encoding $\mathbb{E}_y$. Party 1 locally inserts the encodings into a plain cuckoo hash table $T$ with $m\approx 1.5|\mathbb{E}_y|$ slots using the algorithm specified in \sectionref{sec:prelim}. Party 1 samples an injective function $\pi : m\rightarrow m$ such that for the $i$th $e\in \mathbb{E}_y$ and $T[j]=e$, $\pi(j)=i$. That is, $\pi$ defines the mapping from each rows original position in the table $Y$ to its new position in the cuckoo table $T$.

Recall that parties 0 and 1 respectively hold a 2-out-of-2 secret sharing $Y^0,Y^1$ such that $Y=Y^0\oplus Y^1$. 
Party 1 sends $(\textsc{Switch}, \pi)$ to \f{switch} and party 0 sends $(\textsc{Swich}, Y^0)$\footnote{Note that $Y^0$ has $n$ rows while $\pi$ is defined with an input vector containing $m\approx 1.5n$ rows. Party 0 will pad $Y^0$ with $m-n$ rows which contain all zeros. These will be mapped to the empty slots of the cuckoo table by the permutation/switching network.}. In response \f{switch} sends $\hat Y^{0,1}$ to party 1  and $\hat Y^{0,0}$ to party 2. Party 1 defines their output of this phase as $\hat Y^0:=\hat Y^{0,1} \oplus \pi(Y^1)$ and party 2 defines their output as $\hat Y^{1} =\hat Y^{0,0}$.

It is now the case that $\hat Y = \hat Y^0\oplus \hat Y^1$ is a valid secret shared cuckoo hash table of the original table $Y$. In particular, for a given row $Y[i]$ with encoding $e=\mathbb{E}_y[i]$, there exists a $j\in \{h_1(e),h_2(e), h_3(e)\}$ such that  $\hat Y[j] = Y[i]$. Here, the $h_i$ functions are hash functions used to construct the cuckoo table $T$. Another important observation is that $\pi$ is a permutation and therefore the more efficient permutation protocol can be used in place of the universal switching protocol.

\paragraph{Selecting from the Cuckoo Table}

The next phase of the protocol is to select the appropriate rows of $\hat Y$ and compare them with each row of $X$. This is achieved with party 0s knowledge of the randomized encodings $\mathbb{E}_x$. Namely, party 0 knows that if the joined on columns of the $i$th row $X[i]$ will match with a row from $Y$, then this row will be at $\hat Y[j]$ for some $j\in \{h_1(e),h_2(e), h_3(e)\}$ where  $e=\mathbb{E}_x[i]$. 

To obliviously compare these rows, party 0 will construct three switching networks with programming $\pi_1,\pi_2,\pi_3 : n\rightarrow m$ such that if $h_l(\mathbb{E}_x[i])=j$ then $\pi_l(i)=j$. Each of these will be used to construct a secret shared table $\share{Y^1},\share{Y^2},\share{Y^3}$ which are the result of applying the switching networks $\pi_1,\pi_2,\pi_3$ to $\share{\hat Y}$. In particular, for the $i$th row $X[i]$ it is now the case that if $X[i]$ has a matching row in $Y$ then it will be contained at  $Y^1[i],Y^2[i]$ or ${Y^3}[i]$. 

One in constructing $\pi_1,\pi_2,\pi_3$ is in the case that there is a collision among $\{h_1(e),h_2(e), h_3(e)\}$ where $e= \mathbb{E}_x[i]$. In this case, party 1 will map two some row in $\hat Y$ to the $i$ row of the tables $\share{Y^1},\share{Y^2},\share{Y^3}$. As such, when we compare $X[i]$ with these rows there will be two (or more) matches and the comparison circuit will not know which to select. We overcome this by padding $\{h_1(e),h_2(e), h_3(e)\}$ to the desired size with arbitrary values from $[m]$. Therefor it is guaranteed that for the $i$th row, the joined on columns of $\{\share{Y^1}[i],\share{Y^2}[i],\share{Y^3}[i]\}$ will  have at most one intersection with $X[i]$.


\paragraph{Inner Join}

Given the four secret shared tables $\share{X},\share{Y^1},\share{Y^2},\share{Y^3}$ as described above, the parties do a learn pass over the $n$ rows to construct the inner join between $X$ and $Y$. Recall that the inner join consists of all the selected columns from the concatenated rows $X[i],Y[j]$ where  the joined on columns of the rows $X[i]$ and $Y[j]$ are equal. Alternatively, an inner join can be thought of as the intersection between the joined on columns of $X$ and $Y$. 

By construction, if row $X[i]$ has a matching row in $Y$ then this row will occupy at most one of the rows ${Y^1}[i],{Y^2}[i],{Y^3}[i]$. To determine which the parties input the secret shared of these rows to an MPC protocol where the join on columns are compared. In particular, for each $i$ and rows ${Y^1}[i],{Y^2}[i],{Y^3}[i]$ the bits $b_1[i],b_2[i],b_3[i]$ are generated where $b_l[i]=1$ iff the joined on columns of ${Y^1}[i]$ are equal to that of $X[i]$. The MPC circuit then computes $b[i]=b_1[i]\oplus b_2[i]\oplus b_3[i]$ and $Y'[i]=b_1[i]{Y^1}[i]\oplus b_2[i]{Y^2}[i]\oplus b_3[i]{Y^3}[i]$. $b[i]$ encodes whether the $i$th row of the output table is valid. If it is valid, then $Y'[i]$ is the row of $Y$ which matches $X[i]$.

The next phase of the join protocol is to compute the \texttt{where} clause of the query and any addition computation which is specified by the \texttt{select} clause. The \texttt{where} clause further filters the output table as a function of $Y'[i]$ and $X[i]$. For example, the query may specify that only rows where $Y_2'[i] + X_3[i] > 22$  are to be select. Regardless of the exact where clause, the MPC protocol will update the valid bit as $b[i] := b[i] \wedge f(Y'[i], X[i])$ where $f: \mathcal{Y}\times \mathcal{X} \rightarrow \{0,1\}$ is the filtering function specified by the \texttt{where} clause.

Finally, the additional computation specified by the \texttt{select} query is performed. Specifically, the columns of the output table can either be directly copied from the input tables $X,Y$ or can be a function of the given row. For example, the query my be of the form 
$$
\texttt{select } X_1, Y_2 + X_3 \text{ from } X \texttt{ inner join } Y \texttt{ on } X_1 = Y_1
$$
In this case the first column of the output will be $X_1$ while the second column will consist of the second column of $Y$ plus the third column of $X$. In general we can view the \texttt{select} clause as a function which takes the two rows $X[i]$ and $Y'[i]$ and computes a new row with the specified output columns. 

Several optimizations can be applied to this protocol. First, observe that only columns of $Y$ which explicitly appear in the query need to be input to the switching networks. This reduces the amount of data to be sent and improves performance. Secondly, when comparing the joined on columns, instead of computing the equality circuit between all of these columns it suffices to compare the randomized encodings. In the event that the joined on column(s) contain many bits, comparing the encodings can reduce the size of the equality circuit. In addition, observe that including columns from $X$ in the output table is essentially free due to these secret share columns simply being copied from $X$. Leveraging this the queries can be optimized by ensuring that the majority of the output columns are taken from $X$. Moreover, if a joined on column from $Y$ is in the \texttt{select} clause, this output column can be replaced with the matching column in $X$.

Also observe that the computation perform heavily lends itself to SIMD instructions. That is, the same computation is repeatedly applied to each row of the output table. Modern MPC protocol such as the ABY$^3$ framework \cite{aby3,highthroughput} are optimized for this setting and can process billions of binary circuit gates per second\cite{highthroughput}. In addition the ABY$^3$ framework can switch between using binary and arithmetic circuits based on which is most efficient for the given computation. 

\paragraph{Left/Right Join}

A left join query is similar to an inner join except that all of the rows from the left table $X$ are included in the output. In this case all of the output fields which are a function of the right table but do not have a matching row from the right table are initialized as a default value, typically \texttt{NULL}.  This type of join has the property that all of the output rows maintain their valid\footnote{More specifically, the output rows copy the validity bit $b[i]$ from the $X$ table.} and the valid bits $b[i]$ need not be updated. Instead the bits $b[i]$ as computed in the inner join will be used to determine if the fields that are a function of $Y$ need to be initialized to the default value. That is, all rows that are in the inner join are computed as before and the remaining rows from $X$ use the bit $b[i]$ to initialize the missing values to the default. A right join can be implemented by swapping $X$ an $Y$.

\paragraph{Union and Set Minus}

Our framework is also capable of computing the union of two table. Specifically, we define the union operator as taking all of the rows from the left table and all of the rows from the right table which would not be present in the inner join. Note that the union operator assumes that the schemes of the two tables are identical. Somewhat surprisingly the protocol implementing union is conceptually similar to the inner join. The key observation is that the inner join computes a valid bit for each row of $X$ which encodes whether that row is in the intersection. By flipping that bit\footnote{Note that the special case where a row of $X$ is not valid as opposed to being outside the intersection must be handled.}, it now encodes whether that row is in $X\backslash Y$. Given such an operator the union of $X$ and $Y$ can trivially be constructed as $(X\backslash Y) || Y$ where the $||$ operator denotes the row-wise concatenation of $Y$ to the end of $X\backslash Y$.


\paragraph{Full Join}

A full join can be viewed as a hybrid between an inner join and a union. Specifically, the output table contains the inner of $X$ and $Y$ which is then unioned with $X$ and $Y$ on the right side. Put another way, a full join consists of the inner join where the missing rows from $X,Y$ additionally included. This join is implemented using two other joins. First a left join between $X$ and $Y$ is performed. As a result the rows in the inner join are correctly compute and all records from $X$ are contained in the output table. What remains is to add all of the missing rows from $Y$ which is achieved by taking the union with the output table and $Y$. The overhead of this protocol is effectually twice that of the other protocol.

\subsection{Cardinality Revealing Joins}


\subsection{Non-unique Join on Column}

\section{Beyond Three Parties}

\section{Voter Registration}

\section{Thread Log Comparison}



